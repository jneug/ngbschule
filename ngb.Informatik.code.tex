
%% -----------------
%%
%% Makros für Dokumente im Fach Informatik
%% Vor allem Eindeutschung und Kuerzel für Struktogramm und UML Makros
%%
%% -----------------

%% Weitere Informatik-Pakete
\newenvironment{klassenDoku}{%
	\tabularx{\textwidth}{|l|X|}\hline
		\rowcolor{ngb.tabelle.kopf.hg}
		\textbf{Methode} & \textbf{Beschreibung} \\ \hline
}{\endtabularx}
\def\methodenDoku#1#2{%
	\texttt{\textbf{#1}} & #2 \\ \hline
}

%% Styling von Quelltexten
\lstset{
	basicstyle=\small\ttfamily,
	keywordstyle=\color{black}\bfseries,
	identifierstyle=, % nothing happens
	commentstyle=\color{gray},
	stringstyle=,
	showstringspaces=false,
	numbers=left,
	numberstyle=\ttfamily\scriptsize\color{black!75},
	breaklines=true,
	tabsize=4,
}

% Sprache LibreLogo für Quelltexte einfügen
\lstdefinelanguage{LibreLogo}[]{Logo}{%
	morekeywords={vor,vr,zurück,zk,links,li,rechts,re,fliegen,laufen,position,pos,richtung,ri,verbergen,zeigen,anfang,säubern,füllen,schliessen,stiftbreite,linienbreite,sb,lb,stiftfarbe,linienfarbe,sf,lf,stifttransparenz,linientransparenz,st,lt,stiftende,linienende,se,le,stiftübergang,linienübergang,sü,lü,stiftstil,linienstil,ss,ls,füllfarbe,ff,fülltransparenz,ft,füllstil,fs,kreis,ellipse,quadrat,rechteck,punkt,schreibe,text,schriftfarbe,textfarbe,schf,tf,schriftart,scha,schriftgrösse,textgrösse,schg,tg,schriftgewicht,schg,schriftstil,schs},%
	morekeywords=[2]{wiederhole,wdh,zu,ende,wenn,bild,für,in,solange,abbruch,weiter,zähler,und,nicht,oder,ausgabe,eingabe,rückgabe,zufällig,stopp,warte,global,ganz,dezimal,zeichen,wurzel,sin,cos,log10,runde,zähle,menge,folge,liste,tupel,sortiert,ersetzt,suche,findealle,min,max},%
	morekeywords=[3]{wahr,falsch,seite,beliebig,bel,pi},%
	sensitive=false,
	morecomment=[l]{;},
	morestring=[b]",
}[keywords]
\lstdefinestyle{LibreLogo}{
	language={LibreLogo},
	showspaces=false,
	showstringspaces=false,
	showtabs=false,
	tabsize=3,
	extendedchars=true,
	basicstyle=\ttfamily\small,		keywordstyle={\color{blue}\underbar},
	identifierstyle=,
	commentstyle=\color{gray},
	backgroundcolor=\color{white},
	numbers=left,
	numberstyle=\sffamily\tiny\color{gray},
	stepnumber=1,
	numbersep=5pt,
	captionpos=b,
	breaklines=true
}

\def\zeilennummernAus{\lstset{numbers=none}}
\def\zeilennummernEin{\lstset{numbers=left}}
\def\setzeSprache#1{\lstset{language=#1}}

% Struktogramme (struktex)
\sBoolValue{\texttt{true}}{\texttt{false}}
\newcommand{\wenndann}[4][8]{\ifthenelse[#1]{#2}{#3}{#4}{\sTrue}{\sFalse}}
\newcommand{\anweisung}[2][8]{\assign[#1]{#2}}
\newcommand{\leer}[1][8]{\assign[#1]{}}
\def\sonst{\change}
\def\wennende{\ifend}

% UML (pgf-umlcd)
\newenvironment{klassendiagramm}{\begin{tikzpicture}}{\end{tikzpicture}}

%% Notizen/Templates für Inhaltsobjekte!
\def\attribut{\attribute}
\def\methode{\operation}
\newenvironment{klasse}[3][text]{\begin{class}[#1]{#2}{#3}}{\end{class}}
\newenvironment{objekt}[3][text]{\begin{object}[#1]{#2}{#3}}{\end{object}}
\def\benutzt{\unidirectionalAssociation}
% Selbstassoziation
% Von: https://tex.stackexchange.com/questions/98021/how-to-extend-pgf-umlcd-with-self-association-connection#98023
\newcommand{\benutztSelbst}[3]{%
\coordinate (a) at ($(#1.east) + (0,1)$);
\coordinate (b) at ($(#1.east) + (1,1)$);
\coordinate (d) at ($(#1.east) + (1,-1)$);
\coordinate (e) at ($(#1.east) + (0,-1)$);
\coordinate (t) at ($(#1.east) + (1,0)$);
\coordinate (c) at ($(d)!(b)!(t)$);
  \draw [umlcd style,<-] (a) -- (b)
  node[midway, above]{#2}
  node[midway, below]{#3};
  \draw [umlcd style] (b) -- (c);
  \draw [umlcd style] (c) -- (d);
  \draw [umlcd style] (d) -- (e);
}

%% Laden von vordefinierten Struktogrammen, 
%% Klassendiagrammen und Quelltexten aus der internen
%% Bibliothek
\def\ngb@defaultPath{inc/if}
\def\ngb@defaultLang{java}
\newcommand{\nss}[1]{\centernssfile{\ngb@defaultPath/#1}}
\newcommand{\cd}[1]{\input{\ngb@defaultPath/#1.cd.tex}}
\newcommand{\lst}[1]{\IfFileExists{\ngb@defaultPath/#1.lst.tex}%
	{\input{\ngb@defaultPath/#1.lst.tex}}%
	{\input{\ngb@defaultPath/#1.\ngb@defaultLang.lst.tex}%
}}
% PAPs
% Syntaxdiag

%% Hack für Listings in tabularx
\newcommand{\tablelst}[2][]{\lstinline[#1]{#2}}
\newenvironment{savelst}[1]
{\expandafter\newbox\csname lstbox#1\endcsname\lrbox{\csname lstbox#1\endcsname}}
{\endlrbox}
\newcommand{\loadlst}[1]{\usebox{\csname lstbox#1\endcsname}}

%\newcolumntype{P}[2]{>{\Zeilenhoehe[#1]}p{#2}}
%\newcolumntype{M}[2]{>{\Zeilenhoehe[#1]}m{#2}}
%\newcolumntype{B}[2]{>{\Zeilenhoehe[#1]}b{#2}}
%\newcolumntype{Y}[1]{>{\Zeilenhoehe[#1]}X}