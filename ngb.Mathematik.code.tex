
% Mathe-Pakete
\RequirePackage{amsmath,amssymb,amstext,amscd}
\RequirePackage{exscale}
\RequirePackage{tikz}

% Kartesische Koordinatengeometrie und Funktionen
\RequirePackage{tkz-base,tkz-euclide,tkz-fct} % requires gnuplot
\usetkzobj{all}
%\usetikzlibrary{shapes}
%\usetikzlibrary{calc}
%\RequirePackage{xstring}

\RequirePackage[commonsets,notation=german]{skmath}
%\RequirePackage{units}
\RequirePackage{gauss}
\RequirePackage{interval}
\intervalconfig{
	separator symbol = {;\,},
}


% Schnell-Kommandos für Mengenzeichen und andere Symbole
\newcommand{\bs}{\,\backslash\,}
\newcommand{\qed}{\ensuremath{\Box}}

\newenvironment{sachaufgabe}{\begin{description}}{\end{description}}
\newcommand{\frage}{\item[Frage:]}
\newcommand{\rechnung}{\item[Rechnung:]}
\newcommand{\antwort}{\item[Antwort:]}

%\newcommand{\grad}[1]{\ensuremath{\unit[#1]{^\circ}}}
\def\grad{\ang}

\newcommand{\punkt}{\@ifstar\@@punkt\@punkt}
\newcommand{\@punkt}[2]{\ensuremath{\left(#1 \middle| #2\right)}}
\newcommand{\@@punkt}[3]{\ensuremath{\left(#1 \middle| #2 \middle| #3\right)}}

% https://tex.stackexchange.com/questions/249688/how-do-i-draw-a-circle-around-a-term-in-an-align-equation


\def\geogrid{5mm}
\def\geogridcolor{gray!50}
\def\geoaxiscolor{black}
\def\georeset{%
	\renewcommand{\geogrid}{5mm}%
	\renewcommand{\geogridcolor}{gray!50}%
	\renewcommand{\geoaxiscolor}{black}%
}

\tikzset{
	crossing/.style={
		draw,
		cross out,
		minimum size=2mm,
		inner sep=0pt,
		outer sep=0pt,
		thick
	}
}


%%%%%%%%%%%%%%%%%%%%%%%%%%%%%%%%%%%%%%%%%%%%
%                Umgebungen                %
%%%%%%%%%%%%%%%%%%%%%%%%%%%%%%%%%%%%%%%%%%%%
\newenvironment{koordinatensystem}[5][1cm]{%
	\begin{tikzpicture}[x=#1, y=#1, smooth]
	% Setup tkz
	\tkzInit[xmin=#2,xmax=#3,ymin=#4,ymax=#5]
	\tkzClip[space=1]
	\tkzGrid[color=gray!50,step=#1]
	\tkzAxeX[below right,label=$x$]  
	\tkzAxeY[above left,label=$y$]
	\tkzClip[space=.5]
}{\end{tikzpicture}}

\newenvironment{koordinatensystem*}[1]{%
	\begin{tikzpicture}[smooth]
	% Setup tkz
	\tkzInit[#1]
	\tkzClip[space=1]
	\tkzGrid[color=gray!50]
	\tkzAxeX[below right,label=$x$]  
	\tkzAxeY[above left,label=$y$]
	\tkzClip[space=.5]
}{\end{tikzpicture}}

\newenvironment{koordinatensystemNO}[3][1cm]{\begin{koordinatensystem}[#1]{0}{#2}{0}{#3}}{\end{koordinatensystem}}

\newenvironment{koordinatensystemN}[4][1cm]{\begin{koordinatensystem}[#1]{#2}{#3}{0}{#4}}{\end{koordinatensystem}}

%\newenvironment{zahlenstrahl}[3][1cm]{%
%	\begin{tikzpicture}[]
%}{\end{tikzpicture}}


%%%%%%%%%%%%%%%%%%%%%%%%%%%%%%%%%%%%%%%%%
%                Objekte                %
%%%%%%%%%%%%%%%%%%%%%%%%%%%%%%%%%%%%%%%%%

\newcommand{\Text}[3][]{\tkzText[fill=white,#1](#2){#3}}

% Punkte %
\newcommand{\Ursprung}[1][O]{\tkzDefPoint(0,0){#1}}
\newcommand{\Koordinate}[2]{\tkzDefPoint(#1){#2}}
%\newcommand{\Punkt}[3][]{\tkzDefPoint(#2){#3}\tkzDrawPoint[size=10,#1](#3)}
\newcommand\Punkt{\@ifstar\@@Punkt\@Punkt}
\newcommand\@Punkt[3][]{\node at (#2) [crossing] (#3) {};\PunktBeschriften{#3}{#3}}
\newcommand\@@Punkt[3][]{\node at (#2) [crossing] (#3) {};}
\newcommand{\PunktMarkieren}[1]{\tkzPointShowCoord(#1)}
\newcommand{\PunktBeschriften}[3][]{\tkzLabelPoint[#1](#2){#3}}

\def\Spiegelpunkt{\@ifstar\@@Spiegelpunkt\@Spiegelpunkt}
\newcommand{\@Spiegelpunkt}[4][]{\tkzDefPointBy[symmetry=center #3](#2)\tkzGetPoint{#4}\tkzDrawPoint[size=10,#1](#4)}
\newcommand{\@@Spiegelpunkt}[4][]{\tkzDefPointBy[reflection=over #3](#2)\tkzGetPoint{#4}\tkzDrawPoint[size=10,#1](#4)}

\def\Drehpunkt{\@ifstar\@@Drehpunkt\@Drehpunkt}
\newcommand{\@Drehpunkt}[5][]{\tkzDefPointBy[rotation=center #3 angle #4](#2)\tkzGetPoint{#5}\tkzDrawPoint[size=10,#1](#5)}
\newcommand{\@@Drehpunkt}[4][]{}

% Linien und Vektoren %

\newcommand{\Strecke}[2][]{\tkzDrawSegment[#1](#2)}
\newcommand{\StreckeBeschriften}[3][]{\tkzLabelSegment[#1](#2){#3}}

\newcommand{\Gerade}[2][]{\tkzDrawLine[add=10 and 10,#1](#2)}
\newcommand{\GeradeBeschriften}[3][]{\tkzLabelSegment[#1](#2){#3}}

\newcommand{\Strahl}[2][]{\tkzDrawLine[add=0 and 10,#1](#2)}

\newcommand{\Vektor}[2][]{\Strecke[>=latex,->,thick,#1]{#2}}

% Formen und Polygone %
\newcommand{\Polygon}[2][]{\tkzDrawPolySeg[#1](#2)}

% Transformationen %
\newcommand{\Spiegeln}[3][']{%
	\foreach \i in {#2}{%
		\Spiegelpunkt{\i}{#3}{\i#1}%
	}%
}
%\newcommand{\SpiegelnUndZeichnen}[3][']{\Spiegeln[#1]{#2}{#3}\Polygon{\MitSuffixVersehen{#2}{#1}}}

\newcommand{\MitSuffixVersehen}[2][']{\foreach \i in {#2}{\i#1,}}