
% Mathe-Pakete
\RequirePackage{amsmath,amssymb,amstext,amscd}
\RequirePackage{exscale}
\RequirePackage{tikz}

% Kartesische Koordinatengeometrie und Funktionen
\RequirePackage{tikz}
\RequirePackage{tkz-base,tkz-euclide,tkz-fct} % requires gnuplot
\usetkzobj{all}
%\usetikzlibrary{shapes}
%\usetikzlibrary{calc}
%\RequirePackage{xstring}

\RequirePackage[commonsets,notation=german]{skmath}
%\RequirePackage{units}
\RequirePackage{gauss}
\RequirePackage{interval}
\intervalconfig{
	separator symbol = {;\,},
}
\RequirePackage{longdivision}
\longdivisionkeys{style=german}


% Schnell-Kommandos für Mengenzeichen und andere Symbole
\newcommand{\bs}{\,\backslash\,}
\newcommand{\qed}{\ensuremath{\Box}}

\newenvironment{sachaufgabe}{\begin{description}}{\end{description}}
\newcommand{\frage}{\item[Frage:]}
\newcommand{\rechnung}{\item[Rechnung:]}
\newcommand{\antwort}{\item[Antwort:]}

%\newcommand{\grad}[1]{\ensuremath{\unit[#1]{^\circ}}}
\def\grad{\ang}

\newcommand{\punkt}{\@ifstar\@@punkt\@punkt}
\newcommand{\@punkt}[2]{\ensuremath{\left(#1 \,\middle|\, #2\right)}}
\newcommand{\@@punkt}[3]{\ensuremath{\left(#1 \,\middle|\, #2 \,\middle|\, #3\right)}}

\def\pkt{\@ifstar\@@pkt\@pkt}
\def\@pkt(#1|#2){\ensuremath{\left(#1 \,\middle|\, #2\right)}}
\def\@@pkt(#1|#2|#3){\ensuremath{\left(#1 \,\middle|\, #2 \,\middle|\, #3\right)}}


\NewDocumentCommand \gls {s O{\rightsquigarrow} m O{.8em}} {
	\IfBooleanTF{#1}{
		\begin{gmatrix}[v]
		#3
		\end{gmatrix}
	}{
		#2& \begin{gmatrix}[v]
		#3
		\end{gmatrix}\\[#4]
	}
}
\NewDocumentCommand \kom {s O{\rightsquigarrow} m O{.8em}} {
	\IfBooleanTF{#1}{
		\begin{gmatrix}[p]
			#3
		\end{gmatrix}
	}{
		#2& \begin{gmatrix}[p]
			#3
		\end{gmatrix}\\[#4]
	}
}

\NewDocumentCommand \gl {O{\Leftrightarrow} m O{=} m} {
	#1\quad&& #2 &#3 #4 && 
}
\NewDocumentCommand \tu {o d() o d() o} {
	\qquad\vert \IfNoValueTF{#1}{%
		\,\text{\IfNoValueTF{#2}{TU}{#2}}}{%
		#1\IfNoValueF{#2}{\quad\vert\,\text{#2}}}%
	\IfNoValueF{#3}{\quad\vert #3}%
	\IfNoValueF{#4}{\quad\vert\,\text{#4}}%
	\IfNoValueF{#5}{\quad\vert #5}%
}

% https://tex.stackexchange.com/questions/249688/how-do-i-draw-a-circle-around-a-term-in-an-align-equation


% Subgrid farbe
\def\tkzCoeffSubColor{25}
\def\tkzCoeffSubLw{0.2}

% Einstellungen für Subgrid
%\def\geogrid{5mm}
%\def\geogridcolor{gray!50}
%\def\geoaxiscolor{black}
%\def\georeset{%
%	\renewcommand{\geogrid}{5mm}%
%	\renewcommand{\geogridcolor}{gray!50}%
%	\renewcommand{\geoaxiscolor}{black}%
%}

\tikzset{
	crossing/.style={%
		draw,
		cross out,
		minimum size=2mm,
		inner sep=0pt,
		outer sep=0pt,
		thick
	},
	%	point style/.style={%
	%		shape=cross out
	%	}
}

%%%%%%%%%%%%%%%%%%%%%%%%%%%%%%%%%%%%%%%%%%%%
%                Umgebungen                %
%%%%%%%%%%%%%%%%%%%%%%%%%%%%%%%%%%%%%%%%%%%%
\newenvironment{koordinatensystem}[5][1cm]{%
	\begin{tikzpicture}[x=#1, y=#1, smooth]
	% Setup tkz
	\tkzInit[xmin=#2,xmax=#3,ymin=#4,ymax=#5]
	\tkzClip[space=1]
	\tkzGrid[color=gray!50,step=#1]
	\tkzAxeX[below right,label=$x$]  
	\tkzAxeY[above left,label=$y$]
	\tkzClip[space=.5]
}{\end{tikzpicture}}

\newenvironment{koordinatensystem*}[1]{%
	\begin{tikzpicture}[smooth]
	% Setup tkz
	\tkzInit[#1]
	\tkzClip[space=1]
	\tkzGrid[color=gray!50]
	\tkzAxeX[below right,label=$x$]  
	\tkzAxeY[above left,label=$y$]
	\tkzClip[space=.5]
}{\end{tikzpicture}}

\newenvironment{koordinatensystemNO}[3][1cm]{\begin{koordinatensystem}[#1]{0}{#2}{0}{#3}}{\end{koordinatensystem}}

\newenvironment{koordinatensystemN}[4][1cm]{\begin{koordinatensystem}[#1]{#2}{#3}{0}{#4}}{\end{koordinatensystem}}

%\newenvironment{zahlenstrahl}[3][1cm]{%
%	\begin{tikzpicture}[]
%}{\end{tikzpicture}}

\newcommand{\geoLinienbreite}[1][1.2pt]{%
\tikzset{
	line style/.style={line width=#1},
	every path/.append style={line width=#1}
}}
\newcommand{\geoPunktform}[1][cross out]{\tikzset{point style/.style={shape=#1}}}

\newcommand{\geoInit}[1][]{\tkzInit[#1]}
\newcommand{\geoGitter}[1][]{\tkzGrid[#1]}
\newcommand{\geoAxen}[1][]{\tkzAxeXY[#1]}

%%%%%%%%%%%%%%%%%%%%%%%%%%%%%%%%%%%%%%%%%
%                Objekte                %
%%%%%%%%%%%%%%%%%%%%%%%%%%%%%%%%%%%%%%%%%

% Text mittig am Punkt #2
\NewDocumentCommand \geoText { O{} r() m } {\tkzText[#1](#2){#3}}

% Punkte
\newcommand{\geoUrsprung}[1][O]{\tkzDefPoint(0,0){#1}}
\NewDocumentCommand \geoKoordinate { r() m }{\tkzDefPoint(#1){#2}}
\NewDocumentCommand \geoKoordinaten {m}{\tkzDefPoints{#1}}

\NewDocumentCommand \geoPunkt {s O{} r() m}{
	\tkzDefPoint(#3){#4}\tkzDrawPoint[shape=cross out,size=10,#2](#4)
	\IfBooleanT {#1}
	{ \geoPunktBeschriften(#4){#4} }
}

\NewDocumentCommand \geoPunktMarkieren { r() }{\tkzPointShowCoord(#1)}

\NewDocumentCommand \geoPunktBeschriften { O{} r() m }{\tkzLabelPoint[#1](#2){#3}}

% TODO: Spiegelpuntke auch shape=cross out setzen, oder als Kreis lassen?
\NewDocumentCommand \geoSpiegelpunkt {s O{} r() r() m}{
	\IfBooleanTF {#1}
	{\tkzDefPointBy[reflection=over #4](#3)\tkzGetPoint{#5}\tkzDrawPoint[size=10,#2](#5)}
	{\tkzDefPointBy[symmetry=center #4](#3)\tkzGetPoint{#5}\tkzDrawPoint[size=10,#2](#5)}
}

\NewDocumentCommand \geoDrehpunkt {s O{} r() r() m m}{
	\IfBooleanTF {#1}
	{\tkzDefPointBy[rotation=center #4 angle #5](#3)\tkzGetPoint{#6}\tkzDrawPoint[size=10,#2](#6)}
	{\tkzDefPointBy[rotation=center #4 angle {-1*#5}](#3)\tkzGetPoint{#6}\tkzDrawPoint[size=10,#2](#6)}
}

% Linien und Vektoren %

\NewDocumentCommand \geoStrecke { O{} r() }{\tkzDrawSegment[#1](#2)}
\NewDocumentCommand \geoStrecken { O{} r() }{\tkzDrawSegments[#1](#2)}
\NewDocumentCommand \geoStreckeBeschriften { O{} r() m }{\tkzLabelSegment[#1](#2){#3}}

\NewDocumentCommand \geoLaenge { O{} r() O{} m } {\tkzDrawSegment[|-|,#1](#2)\tkzLabelSegment[#3](#2){#4}}

% Args:
% location: below, above, left, right
% options for the line
% line endings
% options for the label
% label text
\NewDocumentCommand \geoAbmessung { m O{} r() O{} m } {%
	\pgfmathsetmacro \xd {0}
	\pgfmathsetmacro \yd {0}
	\ifstrequal{#1}{right}{\pgfmathsetmacro \xd {2mm}}{}
	\ifstrequal{#1}{left}{\pgfmathsetmacro \xd {-2mm}}{}x
	\ifstrequal{#1}{above}{\pgfmathsetmacro \yd {2mm}}{}
	\ifstrequal{#1}{below}{\pgfmathsetmacro \yd {-2mm}}{}
	\begin{scope}[transform canvas={xshift=\xd,yshift=\yd}]
	 	\geoLaenge[#2](#3)[#1,#4]{#5}
	\end{scope}%
}

\NewDocumentCommand \geoGerade { O{} r() }{\tkzDrawLine[add=10 and 10,#1](#2)}
\NewDocumentCommand \geoGeradeBeschriften { O{} r() m }{\tkzLabelSegment[#1](#2){#3}}

\NewDocumentCommand \geoStrahl { O{} r() }{\tkzDrawLine[add=0 and 10,#1](#2)}

\NewDocumentCommand \geoVektor { O{} r() }{\tkzDrawSegment[>=latex,->,thick,#1](#2)}

% Formen und Polygone %
%\NewDocumentCommand \geoPolygon { O{} r() }{\tkzDrawPolySeg[#1](#2)}
\NewDocumentCommand \geoPolygon { O{} r() }{\tkzDrawPolygon[#1](#2)}

\NewDocumentCommand \geoPolygonzug { O{} m }{\draw[#1] #2;}

%\newcommand{\geoWuerfel}[4]{
%	\pgfmathsetmacro \angle {45}
%	\pgfmathsetmacro \xd {{.5*cos(\angle)}}
%	\pgfmathsetmacro \yd {{.5*sin(\angle)}}
%	\pgfmathsetmacro \x {{#1-1+(#2-1)*(\xd)}}
%	\pgfmathsetmacro \y {{#3-1+(#2-1)*(\yd)}}
%	
%	\draw[fill=#4] (\x,\y) -- (\x+1,\y) -- (\x+1,\y+1) -- (\x,\y+1) -- cycle;
%	\draw[fill=#4] (\x,\y+1) -- (\x+\xd,\y+1+\yd) -- (\x+1+\xd,\y+1+\yd) -- (\x+1,\y+1) -- cycle;
%	\draw[fill=#4] (\x+1,\y+1) -- (\x+1+\xd,\y+1+\yd) -- (\x+1+\xd,\y+\yd) -- (\x+1,\y) -- cycle;
%}
\newcommand{\geoWuerfel}[4]{
	\geoWuerfelVorne{#1}{#2}{#3}{#4}
	\geoWuerfelOben{#1}{#2}{#3}{#4}
	\geoWuerfelRechts{#1}{#2}{#3}{#4}
}
\newcommand{\geoWuerfelVorne}[4]{
	\pgfmathsetmacro \xd {.5}
	\pgfmathsetmacro \yd {.5}
	\pgfmathsetmacro \x {{#1-1+.5*(#2-1)}}
	\pgfmathsetmacro \y {{#3-1+.5*(#2-1)}}
	\draw[fill=#4] (\x,\y) -- (\x+1,\y) -- (\x+1,\y+1) -- (\x,\y+1) -- cycle;
}
\newcommand{\geoWuerfelOben}[4]{
	\pgfmathsetmacro \xd {.5}
	\pgfmathsetmacro \yd {.5}
	\pgfmathsetmacro \x {{#1-1+.5*(#2-1)}}
	\pgfmathsetmacro \y {{#3-1+.5*(#2-1)}}
	\draw[fill=#4] (\x,\y+1) -- (\x+\xd,\y+1+\yd) -- (\x+1+\xd,\y+1+\yd) -- (\x+1,\y+1) -- cycle;
}
\newcommand{\geoWuerfelRechts}[4]{
	\pgfmathsetmacro \xd {.5}
	\pgfmathsetmacro \yd {.5}
	\pgfmathsetmacro \x {{#1-1+.5*(#2-1)}}
	\pgfmathsetmacro \y {{#3-1+.5*(#2-1)}}
	\draw[fill=#4] (\x+1,\y+1) -- (\x+1+\xd,\y+1+\yd) -- (\x+1+\xd,\y+\yd) -- (\x+1,\y) -- cycle;
}
% Transformationen %
\NewDocumentCommand \geoSpiegeln { s r() r() O{'} }{%
	\IfBooleanTF {#1} {%
		\foreach \p in {#2}{%
			\geoSpiegelpunkt*(\p)(#3){\p#4}
		}%
	}{%
		\foreach \p in {#2}{%
			\geoSpiegelpunkt(\p)(#3){\p#4}
		}%
	}
}
%\newcommand{\geoSpiegelnUndZeichnen}[3][']{\Spiegeln[#1]{#2}{#3}\Polygon{\MitSuffixVersehen{#2}{#1}}}

\newcommand{\geoMitSuffixVersehen}[2][']{\foreach \i in {#2}{\i#1,}}