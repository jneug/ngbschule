\RequirePackage{scrlayer-scrpage}
\pagestyle{scrheadings}

\newcommand{\schule@dokumentTypBezeichnung}{Sitzplan}

\RequirePackage{tikz}
\usetikzlibrary{matrix,positioning,shadows}

\ihead{}\ifoot{}
\chead{}\cfoot{}
\ohead{}\ofoot{}

\newcommand{\ngb@splanihead}{%
\begin{tikzpicture}
\node[draw,fill=black!20,rounded corners=1mm,align=left,inner sep=2mm,text width=4cm,drop shadow]{\large\Titel \\ \Kurs (\Kuerzel)};
\end{tikzpicture}
}
\newcommand{\ngb@splanohead}{%
\begin{tikzpicture}
\node[draw,fill=black!20,rounded corners=1mm,align=left,inner sep=2mm,text width=6cm,drop shadow]{\large Datum: \\Zeit:};
\end{tikzpicture}
}

\ihead{\ngb@splanihead}
\ohead{\ngb@splanohead}
\chead{\DokumentTyp}



\newlength{\tsize}
\setlength{\tsize}{2.4cm}

\newcounter{tischnummer}



\tikzstyle{name label}=[minimum height=\tsize,text width=1.2\tsize,align=center,font={\bfseries},anchor=north west]

\tikzstyle{table}=[draw,line width=2pt,rectangle,minimum width=1.2\tsize,minimum height=\tsize,anchor=north west]

\tikzstyle{htable}=[table,minimum width=2.4\tsize]

\tikzstyle{vtable}=[table,minimum height=2.4\tsize]

\tikzstyle{seating}=[node distance=0,outer sep=0, inner sep=0]

\tikzstyle{tische}=[column sep=2mm, row sep=2mm]

\NewDocumentEnvironment {sitzplan} {} {%
\begin{center}\begin{tikzpicture}[seating]
}{
\end{tikzpicture}\end{center}
}

\NewDocumentCommand \tisch {s m} {\stepcounter{tischnummer}%
	\IfBooleanTF{#1}
		{\node[table,xshift=\tsize] (t\thetischnummer) []{};}
		{\node[table] (t\thetischnummer) []{};}
	\node[name label] (t\thetischnummer-n1) [] at (t\thetischnummer.north west){#2};%
	}
\NewDocumentCommand \htisch {m m} {\stepcounter{tischnummer}%
	\node[htable] (t\thetischnummer) []{};
	\node[name label] (t\thetischnummer-n1) [] at (t\thetischnummer.north west){#1};
	\node[name label,right=of t\thetischnummer-n1] (t\thetischnummer-n2) []{#2};%
}
\NewDocumentCommand \vtisch {m m} {\stepcounter{tischnummer}%
	\node[vtable] (t\thetischnummer) []{};
	\node[name label] (t\thetischnummer-n1) [] at (t\thetischnummer.north west){#1};
	\node[name label,below=of t\thetischnummer-n1] (t\thetischnummer-n2) []{#2};%
}
\NewDocumentCommand \pult {} {
	\node[htable,fill=black!40] (pult) []{\bfseries Pult};%
}
