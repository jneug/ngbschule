\newcommand{\Dauer}{\makeatletter\ngb@dauer\makeatother\xspace}
\newcommand{\Variante}{\makeatletter\ngb@variante\makeatother\xspace}

\graphicspath{{inc/animal-icons/}}
\newcommand{\tier}{\raisebox{-.3\height}{\pgfmathrandom{1,46}\includegraphics[height=1.5cm]{\pgfmathresult.png}}}
\newcommand{\vielErfolg}{\begin{flushright}\bfseries Viel Erfolg \; \tier\end{flushright}}


\newcommand{\TestTitel}[1][none]{\begin{center}\usekomafont{reihe}{\DokumentTyp (\Dauer Minuten)}\\[-1ex]
\usekomafont{title}\Titel\ifstrequal{#1}{none}{}{- #1}\titlerule\end{center}}

\ihead{\small\Fach \Lerngruppe\ifthenelse{\equal{\ngb@variante}{}}{}{- \Variante} (\Kuerzel)}
\chead{\small\Datum}
%\ohead{\DokumentTyp Nr. \DokumentNummer\linebreak\Datum}
\ohead{\small\Namensfeld}

\KOMAoptions{headsepline=off}

%% Überschreiben von aufgabeLaden, um die Variante
%% einzubeziehen. Statt <name>.afg.tex wird zuerst nach
%% <name>-<variante>.afg.tex gesucht.
\renewcommand{\aufgabeLaden}[2][.]{\def\ngb@imgPath{#1}%
	\InputIfFileExists{\ngb@afgPool/#2-\ngb@variante.afg.tex}{}{%
		\InputIfFileExists{\ngb@afgPool/#2.afg.tex}{}{%
			\PackageWarning{ngbschule}{Aufgabendatei \ngb@afgPool/#2.afg.tex wurde nicht gefunden!}%
		}%
}}

% Befehle für A/B-Varianten
\ifdefstring{\ngb@variante}{B}{%
	\def\A#1{}
	\def\B#1{#1}
	\def\AB#1#2{#2}
	\NewEnviron{variA}{}
	\NewEnviron{variB}{\BODY}
}{%
	\def\A#1{#1}
	\def\B#1{}
	\def\AB#1#2{#1}
	\NewEnviron{variA}{\BODY}
	\NewEnviron{variB}{}
}


% Varianten dynamisch in einem Dokument laden
% TODO: Reset der Aufgabennummern
%\newcommand{\ngb@vari}[2]{\if\ngb@variante#1 #2\fi}
%\def\A#1{\ngb@vari{A}{#1}}
%\def\B#1{\ngb@vari{A}{#1}}
%\def\AB#1#2{\ngb@vari{A}{#1}\ngb@vari{B}{#2}}
%\NewEnviron{varianteA}{\ngb@vari{A}{\BODY}}
%\NewEnviron{varianteB}{\ngb@vari{B}{\BODY}}

%\newcommand{\varianteLaden}[2][inhalt]{\def\ngb@variante{#2}%
%\InputIfFileExists{6.4-Arbeit.#1.tex}{}{\textbf{Fehler beim Laden der Inhalte für Variante #2!}}}