\subsection{Dokumentation der Klasse \code{Scanner}}\label{subsec:doc-scanner}

\begin{klassenDokumentation}
	\methodenDokumentation{Konstruktor}{\code{Scanner(InputStream in)}}{
		Erstellt einen neuen Scanner, der Eingaben vom übergebenen \code{InputStream} liest. In der Regel erzeugt man einen Scanner für die Standard-Konsole mit \code{new Scanner(System.in)}.
	}
	\methodenDokumentation{Anfrage}{\code{boolean hasNext()}}{
		Prüft, ob im Scanner eine neue Eingabe vorliegt.
	}
	\methodenDokumentation{Anfrage}{\code{int nextInt()}}{
		Wartet, bis eine neue Eingabe gemacht wurde und versucht sie als Zahl zu lesen. Ist die Eingabe keine Zahl, wird eine \code{InputMismatchException} geworfen.
	}
	\methodenDokumentation{Anfrage}{\code{String nextLine()}}{
		Wartet, bis eine neue Eingabe gemacht wurde, und gibt sie als String zurück.
	}
	\methodenDokumentation{Anfrage}{\code{boolean nextBoolean()}}{
		Wartet, bis eine neue Eingabe gemacht wurde und versucht sie als Wahrheitswert zu lesen. Ist die Eingabe kein Wahrheitswert, wird eine \code{InputMismatchException} geworfen.
	}
	\methodenDokumentation{Anfrage}{\code{double nextDouble()}}{
		Wartet, bis eine neue Eingabe gemacht wurde und versucht sie als Gleitkommazahl zu lesen. Ist die Eingabe keine Gleitkommazahl, wird eine \code{InputMismatchException} geworfen.
	}
\end{klassenDokumentation}