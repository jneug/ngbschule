\ifthenelse{\equal{\schule@klausurtyp}{klausur}}{%
	% Klausuren
	% https://tex.stackexchange.com/questions/223694/how-to-draw-a-text-box-with-shadow-borders-i-have-tried-the-following-but-it-gi#223738
	\newcommand{\KlausurTitel}[1][none]{%
		\begin{tcolorbox}[enhanced,center upper,%
			fontupper=\rmfamily\bfseries,colback=white,%
			drop shadow southeast,sharp corners]
			\Large\Nummer. \TypBezeichnung (\Dauer\,Minuten)\\
			\large\Fach\ \Lerngruppe\ (\Kuerzel)\ifthenelse{\equal{#1}{none}}{}{\\
				\textcolor{gray!50!black}{- #1 -}}
	\end{tcolorbox}}
	\chead{} % Keine Titel in der Kopfzeile
}{
	% Klassen- und Kursarbeiten
	\newcommand{\KlausurTitel}{\begin{center}\LARGE\rmfamily\Titel\end{center}}
	\def\ArbeitTitel{\KlausurTitel}
	
	\ihead{\small\Fach \Lerngruppe\ifthenelse{\equal{\ngb@variante}{}}{}{- \Variante} (\Kuerzel)}
	\chead{\small\Datum}
	%\ohead{\DokumentTyp Nr. \DokumentNummer\linebreak\Datum}
	\ohead{\small\Namensfeld}
	
	\KOMAoptions{headsepline=off}
}

\newcommand{\Dauer}{\makeatletter\ngb@dauer\makeatother\xspace}
\newcommand{\Variante}{\makeatletter\ngb@variante\makeatother\xspace}

\graphicspath{{inc/animal-icons/}}
\newcommand{\tier}{\raisebox{-.3\height}{\pgfmathrandom{1,46}\includegraphics[height=1.5cm]{\pgfmathresult.png}}}
\newcommand{\vielErfolg}{\begin{flushright}\bfseries Viel Erfolg \; \tier\end{flushright}}

%% Anhang in Klausuren/Arbeiten
\RequirePackage{prettyref}
\newrefformat{anhang}{Anhang\,\ref{#1}}
\newcommand\Anhang{\clearpage\appendix\chead{\centering Anhang}}

%% Überschreiben von aufgabeLaden, um die Variante
%% einzubeziehen. Statt <name>.afg.tex wird zuerst nach
%% <name>-<variante>.afg.tex gesucht.
\renewcommand{\aufgabeLaden}[2][\ngb@variante]{%
	\InputIfFileExists{\ngb@afgPool/#2-#1.afg.tex}{}{%
		\InputIfFileExists{\ngb@afgPool/#2.afg.tex}{}{}%
}}
\ifthenelse{\equal{\ngb@variante}{B}}{%
\def\A#1{}
\def\B#1{#1}
\def\AB#1#2{#2}
\NewEnviron{varianteA}{}
\NewEnviron{varianteB}{\BODY}
}{%
\def\A#1{#1}
\def\B#1{}
\def\AB#1#2{#1}
\NewEnviron{varianteA}{\BODY}
\NewEnviron{varianteB}{}
}

%%%%%%%%%%%%%%%%%%%%%%%%%%%%%%%%%%%%%%%
% Eigene Variante der Notenverteilung %
%%%%%%%%%%%%%%%%%%%%%%%%%%%%%%%%%%%%%%%
\IfEqCase{\ngb@notenStil}{
	{kompakt}{\renewcommand{\notenverteilung}{\Bewertungsschema[]}}%
	{ngbschule}{\renewcommand{\notenverteilung}{\Bewertungsschema}}%
	{schule}{}%
}[]

\newcommand{\Bewertungsschema}[1][tendenzen]{
	% Wenn die xsim Umgebung nicht genutzt wurde, werden die gesetzten
	% Punkte jetzt übertragen
	\IfExerciseGoalsSumTF{points}{=0}
	{\AddtoExerciseTypeGoal{aufgabe}{ngbpoints}{24}}
	{}
	\ifthenelse{\boolean{schule@kmkPunkte}}{
		\begin{center}\tiny\begin{tabular}{|l||c|c|c|c|c|c|c|c|c|c|c|c|c|c|c|c|} \hline
				\rowcolor{black!20}
				\textbf{Notenpunkte} & 15 & 14 & 13 & 12 & 11 & 10 & 9 & 8 & 7 & 6 & 5 & 4 & 3 & 2 & 1 & 0 \\ \hline
				\textbf{Schwelle} & 95\% & 90\% & 85\% & 80\% & 75\% & 70\% & 65\% & 60\% & 55\% & 50\% & 45\% & 39\% & 33\% & 27\% & 20\% & 0\% \\ \hline
				\rowcolor{black!10}
				\textbf{Punkte}
				& \schule@punkteZuNote{15} 
				& \schule@punkteZuNote{14} 
				& \schule@punkteZuNote{13} 
				& \schule@punkteZuNote{12} 
				& \schule@punkteZuNote{11} 
				& \schule@punkteZuNote{10} 
				& \schule@punkteZuNote{9} 
				& \schule@punkteZuNote{8} 
				& \schule@punkteZuNote{7} 
				& \schule@punkteZuNote{6} 
				& \schule@punkteZuNote{5} 
				& \schule@punkteZuNote{4} 
				& \schule@punkteZuNote{3} 
				& \schule@punkteZuNote{2} 
				& \schule@punkteZuNote{1} 
				& 0 \\ \hline
		\end{tabular}\end{center}
	}{
		\ifthenelse{\equal{#1}{tendenzen}}{
			\begin{center}\renewcommand{\arraystretch}{1.2}\tiny\begin{tabular}{|l||c|c|c|c|c|c|c|c|c|c|c|c|c|c|c|c|} \hline
					\rowcolor{black!20}
					\textbf{Note} & 1+ & 1 & 1- & 2+ & 2 & 2- & 3+ & 3 & 3- & 4+ & 4 & 4- & 5+ & 5 & 5- & 6 \\ \hline
					\textbf{Schwelle} & 95\% & 90\% & 85\% & 80\% & 75\% & 70\% & 65\% & 60\% & 55\% & 50\% & 45\% & 39\% & 33\% & 27\% & 20\% & 0\% \\ \hline
					\rowcolor{black!10}
					\textbf{Punkte}
					& \schule@punkteZuNote{15} 
					& \schule@punkteZuNote{14} 
					& \schule@punkteZuNote{13} 
					& \schule@punkteZuNote{12} 
					& \schule@punkteZuNote{11} 
					& \schule@punkteZuNote{10} 
					& \schule@punkteZuNote{9} 
					& \schule@punkteZuNote{8} 
					& \schule@punkteZuNote{7} 
					& \schule@punkteZuNote{6} 
					& \schule@punkteZuNote{5} 
					& \schule@punkteZuNote{4} 
					& \schule@punkteZuNote{3} 
					& \schule@punkteZuNote{2} 
					& \schule@punkteZuNote{1} 
					& 0 \\ \hline
			\end{tabular}\end{center}
		}{
			\begin{center}\renewcommand{\arraystretch}{1.1}\small\begin{tabular}{|l||c|c|c|c|c|c|} \hline
					\rowcolor{black!20}
					\textbf{Notenbereich} & 1 & 2 & 3 & 4 & 5 & 6 \\ \hline
					\textbf{ab Schwelle} & 85\% & 70\% & 55\% & 39\% & 20\% & 0\% \\ \hline
					\rowcolor{black!10}
					\textbf{ab Punkte}
					& \schule@punkteZuNote{13}
					& \schule@punkteZuNote{10}
					& \schule@punkteZuNote{7}
					& \schule@punkteZuNote{4}
					& \schule@punkteZuNote{1}
					& 0 \\ \hline
			\end{tabular}\end{center}
		}
	}
}


% Angepasster Bewertungsbogen
\ExplSyntaxOn

\str_new:N \ngb_kompetenzen_str
\bool_new:N \ngb_erwartungen_erste_bool

\RenewDocumentCommand{\erwartung}{m m O{}}{%
	\bool_if:NTF \ngb_erwartungen_erste_bool {
		\bool_gset_false:N \ngb_erwartungen_erste_bool
	}{
		\tl_gput_right:Nn \schule_zeilen_erwartungen_str {\detokenize{\newline}}
	}

	\int_gadd:Nn \schule_aufgaben_punkte_int {\_str_to_int_with_zero:n{#2}} %Punkte
	\int_gadd:Nn \schule_aufgaben_zusatzpunkte_int {\_str_to_int_with_zero:n{#3}} %Zusatzpunkte
%	\tl_gput_right:Nn \schule_zeilen_erwartungen_str {\detokenize{\newline}}
	\tl_gput_right:Nn \schule_zeilen_erwartungen_str {\detokenize{\dots\; #1}}
}

\cs_new:Npn \_ngb_punkte_anzeige:nn #1 #2 {
	\int_compare:nT {#1 > 0}{#1}
	\int_compare:nT {#2 > 0}{\space (#2)}
}

% Zeile für die Aufgabe in der Tabelle setzen (Standard).
\cs_new:Npn \_ngb_aufgaben_erwartungen_zeile {
	\detokenize{\bfseries}
	\GetExerciseProperty{counter}
	\detokenize{& }%
	\schule_zeilen_erwartungen_str%
	\detokenize{& \bfseries}%
	\_ngb_punkte_anzeige:nn {\int_use:N \schule_aufgaben_punkte_int}{\int_use:N \schule_aufgaben_zusatzpunkte_int}
	\detokenize{& \\}%
	\detokenize{\hline}
}

\NewEnviron{kompetenzen}{
%	\str_gclear:N \ngb_kompetenzen_str
	\exp_args:Nno\relax{}{\BODY}
}

% Kompetenzen in Form bringen
\NewDocumentCommand{\kompetenz}{m}{%
	\tl_gput_right:Nn \ngb_kompetenzen_str {\detokenize{#1 & }}
	\tl_gput_right:Nn \ngb_kompetenzen_str {\detokenize{ & \\ \hline}}
}

% Erwartungshorizont (Eine Tabelle für alles)
% --------------------------------------------------------------------
\def\tabularxcolumn#1{m{#1}}
\newcolumntype{M}[1]{>{\large\centering\arraybackslash}m{#1}}
%\newwrite\ehfile

\newcommand{\ngb@erwartungshorizont}{
	%Aufgabenausgabe leeren
	\str_gclear:N \schule_aufgaben_erwartungen_str
	\int_gzero:N \schule_aufgaben_punkte_ges_int
	\int_gzero:N \schule_aufgaben_zusatzpunkte_ges_int
	
	\ForEachUsedExerciseByID{%
		%Variablen für neue Aufgabe neu initialisieren
		\str_gclear:N \schule_zeilen_erwartungen_str
		\int_gzero:N \schule_aufgaben_punkte_int
		\int_gzero:N \schule_aufgaben_zusatzpunkte_int
		\bool_gset_false:N \schule_erwartungen_zeile_gerade_bool
		\bool_gset_true:N \ngb_erwartungen_erste_bool
		
		%Definition der Aufgabe in entsprechende Befehle laden
		\def\ExerciseType{##1}%
		\def\ExerciseID{##2}%
		\GetExercisePropertyTF{erwartungen}{\PropertyValue}{}
		
		%Gesamtaufgabe
		\tl_gput_right:Nx \schule_aufgaben_erwartungen_str \_ngb_aufgaben_erwartungen_zeile
		
		\int_gadd:Nn \schule_aufgaben_punkte_ges_int {\schule_aufgaben_punkte_int} %Punkte
		\int_gadd:Nn \schule_aufgaben_zusatzpunkte_ges_int {\schule_aufgaben_zusatzpunkte_int} %Zusatzpunkte
		
		\_schule_erwartungen_punkte_speichern
	}
	
	%Zusammengebautetes wieder serialisieren
	\tokenize{%
		\schule@aEHCode%
	}{%
		\schule_aufgaben_erwartungen_str
	}
	\tokenize{%
		\ngb@aKECode%
	}{%
		\ngb_kompetenzen_str
	}

	% Aux-Datei öffnen
%	\immediate\openout\ehfile=\jobname.eh.aux

%	\immediate\write\ehfile{\unexpanded{Name: \underline{\Large\hspace{6cm}}}}
	\underline{Name: \Large\hspace{6cm}}
	
	\begin{center}
		\renewcommand{\arraystretch}{1.4}
		\begin{tabularx}{\textwidth}{|M{1cm}|X|M{1.5cm}|M{1.5cm}|} \hline
			% Kopfzeile
			\rowcolor{black!20}
			Afg & Die~Schülerin/der~Schüler\dots & mögl.\newline Punkte & \small erreicht \tabularnewline\hline\hline %
			%Erwartungen
			\schule@aEHCode%
			
			% Fusszeile
			\multicolumn{2}{|r|}{Summe:} & 52 & \tabularnewline\hline
			\multicolumn{2}{|r|}{Prozentual:} &  & \tabularnewline\hline
			\multicolumn{2}{|r|}{Note:} & \multicolumn{2}{c|}{} \tabularnewline\hline
		\end{tabularx}
	\end{center}

	\begin{flushright}
		\underline{\today, \hspace{3cm}}
	\end{flushright}
	
	\notenverteilung
	
	\begin{center}
		\renewcommand{\arraystretch}{1.2}
		\begin{tabularx}{\textwidth}{|p{9cm}|XX|} \hline
			\rowcolor{black!20}
			Kompetenz & \small Kannst~du\newline sehr~gut & \small Musst~du\newline intensiv~üben \\ \hline \hline
			\ngb@aKECode
		\end{tabularx}
	\end{center}
	
	% Erwartungshorizont einfuegen
%	\immediate\closeout\ehfile
%	\input{\jobname.eh.aux}
	\clearpage
}

\renewcommand{\erwartungshorizont}{\ngb@erwartungshorizont}

\ExplSyntaxOff
