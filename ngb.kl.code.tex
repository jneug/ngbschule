\newcommand{\Dauer}{\makeatletter\ngb@dauer\makeatother\xspace}
\newcommand{\Variante}{\makeatletter\ngb@variante\makeatother\xspace}

\pgfmathtruncatemacro\ngb@hmt{\ngb@dauer-\ngb@hmft}
\NewDocumentCommand \Hmft {s} {%
	\IfBooleanTF{#1}{\makeatletter\ngb@hmt\makeatother}{\makeatletter\ngb@hmft\makeatother}\xspace%
}


\graphicspath{{inc/animal-icons/}}
\newcommand{\tier}{\raisebox{-.3\height}{\pgfmathrandom{1,46}\includegraphics[height=1.5cm]{\pgfmathresult.png}}}
\newcommand{\vielErfolg}{\begin{flushright}\bfseries Viel Erfolg \; \tier\end{flushright}}

\ifdefstring{\schule@klausurtyp}{klausur}{% Klausuren
	% https://tex.stackexchange.com/questions/223694/how-to-draw-a-text-box-with-shadow-borders-i-have-tried-the-following-but-it-gi#223738
	\newcommand{\KlausurTitel}[1][none]{%
		\begin{tcolorbox}[enhanced,center upper,%
			fontupper=\rmfamily\bfseries,colback=white,%
			drop shadow southeast,sharp corners]
			\Large\Nummer. \DokumentTyp (\Dauer\,Minuten)\\
			\large\Fach\ \Lerngruppe\ (\Kuerzel)\ifstrequal{#1}{none}{}{\\
				\textcolor{gray!50!black}{- #1 -}}
	\end{tcolorbox}}
	\chead{} % Keine Titel in der Kopfzeile
	
	\renewcommand{\vielErfolg}{\begin{flushright}\bfseries Viel Erfolg\end{flushright}}
	
	\schule@notenschemaSetzen{\ngb@notenschemaB}
}{ % Klassen- und Kursarbeiten
	\newcommand{\KlausurTitel}[1][none]{\begin{center}\LARGE\rmfamily\Titel\ifstrequal{#1}{none}{}{- #1}\end{center}}
	\newcommand{\ArbeitTitel}[1][none]{\KlausurTitel[#1]}
	
	\ihead{\small\Fach \Lerngruppe\ifthenelse{\equal{\ngb@variante}{}}{}{- \Variante} (\Kuerzel)}
	\chead{\small\Datum}
	%\ohead{\DokumentTyp Nr. \DokumentNummer\linebreak\Datum}
	\ohead{\small\Namensfeld}
	
	\KOMAoptions{headsepline=off}
	
	\schule@notenschemaSetzen{\ngb@notenschemaA}
}

\newcommand{\notenschemaSetzen}[1]{\schule@notenschemaSetzen{#1}}

%% Anhang in Klausuren/Arbeiten
\RequirePackage{prettyref}
\newrefformat{anhang}{Anhang\,\ref{#1}}

\newcommand\Anhang{\clearpage\appendix\chead{\centering Anhang}}

%% Überschreiben von aufgabeLaden, um die Variante
%% einzubeziehen. Statt <name>.afg.tex wird zuerst nach
%% <name>-<variante>.afg.tex gesucht.
\renewcommand{\aufgabeLaden}[2][.]{\def\ngb@imgPath{#1}%
	\InputIfFileExists{\ngb@afgPool/#2-\ngb@variante.afg.tex}{}{%
		\InputIfFileExists{\ngb@afgPool/#2.afg.tex}{}{%
			\PackageWarning{ngbschule}{Aufgabendatei \ngb@afgPool/#2.afg.tex wurde nicht gefunden!}%
		}%
}}

% Befehle für A/B-Varianten
\ifdefstring{\ngb@variante}{B}{%
	\def\A#1{}
	\def\B#1{#1}
	\def\AB#1#2{#2}
	\NewEnviron{variA}{}
	\NewEnviron{variB}{\BODY}
	\NewEnviron{varianteA}{\PackageWarning{ngbschule}{Environment varianteA is deprecated for variA.}}
	\NewEnviron{varianteB}{\PackageWarning{ngbschule}{Environment varianteB is deprecated for variB.}\BODY}
}{%
	\def\A#1{#1}
	\def\B#1{}
	\def\AB#1#2{#1}
	\NewEnviron{variA}{\BODY}
	\NewEnviron{variB}{}
	\NewEnviron{varianteA}{\PackageWarning{ngbschule}{Environment varianteA is deprecated for variA.}\BODY}
	\NewEnviron{varianteB}{\PackageWarning{ngbschule}{Environment varianteB is deprecated for variB.}}
}


% Varianten dynamisch in einem Dokument laden
% TODO: Reset der Aufgabennummern
%\newcommand{\ngb@vari}[2]{\if\ngb@variante#1 #2\fi}
%\def\A#1{\ngb@vari{A}{#1}}
%\def\B#1{\ngb@vari{A}{#1}}
%\def\AB#1#2{\ngb@vari{A}{#1}\ngb@vari{B}{#2}}
%\NewEnviron{varianteA}{\ngb@vari{A}{\BODY}}
%\NewEnviron{varianteB}{\ngb@vari{B}{\BODY}}

%\newcommand{\varianteLaden}[2][inhalt]{\def\ngb@variante{#2}%
%\InputIfFileExists{6.4-Arbeit.#1.tex}{}{\textbf{Fehler beim Laden der Inhalte für Variante #2!}}}

%%%%%%%%%%%%%%%%%%%%%%%%%%%%%%%%%%%%%%%
% Eigene Variante der Notenverteilung %
%%%%%%%%%%%%%%%%%%%%%%%%%%%%%%%%%%%%%%%
\ifdefstring{\ngb@notenStil}{kompakt}{\renewcommand{\notenverteilung}{\Bewertungsschema[]}}{}
\ifdefstring{\ngb@notenStil}{ngbschule}{\renewcommand{\notenverteilung}{\Bewertungsschema}}{}

\newcommand{\Bewertungsschema}[1][tendenzen]{
	\ifbool{schule@kmkPunkte}{
		\begin{center}\tiny\begin{tabular}{|l||c|c|c|c|c|c|c|c|c|c|c|c|c|c|c|c|} \hline
				\rowcolor{black!20}
				\textbf{Notenpunkte} & 15 & 14 & 13 & 12 & 11 & 10 & 9 & 8 & 7 & 6 & 5 & 4 & 3 & 2 & 1 & 0 \\ \hline
				\textbf{Schwelle}
				& \GetRelGrade{15}
				& \GetRelGrade{14}
				& \GetRelGrade{13}
				& \GetRelGrade{12}
				& \GetRelGrade{11}
				& \GetRelGrade{10}
				& \GetRelGrade{9}
				& \GetRelGrade{8}
				& \GetRelGrade{7}
				& \GetRelGrade{6}
				& \GetRelGrade{5}
				& \GetRelGrade{4}
				& \GetRelGrade{3}
				& \GetRelGrade{2}
				& \GetRelGrade{1}
				& 0 \\ \hline
				\rowcolor{black!10}
				\textbf{Punkte}
				& \schule@punkteZuNote{15} 
				& \schule@punkteZuNote{14} 
				& \schule@punkteZuNote{13} 
				& \schule@punkteZuNote{12} 
				& \schule@punkteZuNote{11} 
				& \schule@punkteZuNote{10} 
				& \schule@punkteZuNote{9} 
				& \schule@punkteZuNote{8} 
				& \schule@punkteZuNote{7} 
				& \schule@punkteZuNote{6} 
				& \schule@punkteZuNote{5} 
				& \schule@punkteZuNote{4} 
				& \schule@punkteZuNote{3} 
				& \schule@punkteZuNote{2} 
				& \schule@punkteZuNote{1} 
				& 0 \\ \hline
		\end{tabular}\end{center}
	}{
		\ifstrequal{#1}{tendenzen}{
			\begin{center}\renewcommand{\arraystretch}{1.2}\tiny\begin{tabular}{|l||c|c|c|c|c|c|c|c|c|c|c|c|c|c|c|c|} \hline
					\rowcolor{black!20}
					\textbf{Note} & 1+ & 1 & 1- & 2+ & 2 & 2- & 3+ & 3 & 3- & 4+ & 4 & 4- & 5+ & 5 & 5- & 6 \\ \hline
%					\textbf{Schwelle}
%					& \GetRelGrade{15}
%					& \GetRelGrade{14}
%					& \GetRelGrade{13}
%					& \GetRelGrade{12}
%					& \GetRelGrade{11}
%					& \GetRelGrade{10}
%					& \GetRelGrade{9}
%					& \GetRelGrade{8}
%					& \GetRelGrade{7}
%					& \GetRelGrade{6}
%					& \GetRelGrade{5}
%					& \GetRelGrade{4}
%					& \GetRelGrade{3}
%					& \GetRelGrade{2}
%					& \GetRelGrade{1}
%					& 0\,\% \\ \hline
					\rowcolor{black!10}
					\textbf{Punkte}
					& \schule@punkteZuNote{15} 
					& \schule@punkteZuNote{14} 
					& \schule@punkteZuNote{13} 
					& \schule@punkteZuNote{12} 
					& \schule@punkteZuNote{11} 
					& \schule@punkteZuNote{10} 
					& \schule@punkteZuNote{9} 
					& \schule@punkteZuNote{8} 
					& \schule@punkteZuNote{7} 
					& \schule@punkteZuNote{6} 
					& \schule@punkteZuNote{5} 
					& \schule@punkteZuNote{4} 
					& \schule@punkteZuNote{3} 
					& \schule@punkteZuNote{2} 
					& \schule@punkteZuNote{1} 
					& 0 \\ \hline
			\end{tabular}\end{center}
		}{
			\begin{center}\renewcommand{\arraystretch}{1.1}\small\begin{tabular}{|l||c|c|c|c|c|c|} \hline
					\rowcolor{black!20}
					\textbf{Notenbereich} & 1 & 2 & 3 & 4 & 5 & 6 \\ \hline
%					\textbf{ab Schwelle}
%					& \GetRelGrade{13}
%					& \GetRelGrade{10}
%					& \GetRelGrade{7}
%					& \GetRelGrade{4}
%					& \GetRelGrade{1}
%					& 0 \\ \hline
					\rowcolor{black!10}
					\textbf{ab Punkte}
					& \schule@punkteZuNote{13}
					& \schule@punkteZuNote{10}
					& \schule@punkteZuNote{7}
					& \schule@punkteZuNote{4}
					& \schule@punkteZuNote{1}
					& 0 \\ \hline
			\end{tabular}\end{center}
		}
	}
}


% Angepasster Bewertungsbogen
\ExplSyntaxOn

% Berechnung der Notenschwellen anpassbar machen

\renewcommand{\schule@punkteZuNote}[1]
	{ \ifdefstring{\ngb@notenBerechnung}{aufrunden}{\xsim_get_points_for_grade_ceil:n {#1}}{}%
		\ifdefstring{\ngb@notenBerechnung}{abrunden}{\xsim_get_points_for_grade_floor:n {#1}}{}%
		\ifdefstring{\ngb@notenBerechnung}{runden}{\xsim_get_points_for_grade_round:n {#1}}{} }

\cs_new_protected:Npn \xsim_get_points_for_grade_ceil:n #1
{ \prop_get:NnNTF \l__xsim_relative_grades_prop {#1} \l__xsim_tmpa_tl
	{ \fp_to_decimal:n {ceil(\l__xsim_tmpa_tl*\schule_aufgaben_punkte_ges_int)} }
	{ \msg_error:nnn {xsim} {grade-unknown} {#1} }%
}
\cs_new_protected:Npn \xsim_get_points_for_grade_floor:n #1
{ \prop_get:NnNTF \l__xsim_relative_grades_prop {#1} \l__xsim_tmpa_tl
	{ \fp_to_decimal:n {floor(\l__xsim_tmpa_tl*\schule_aufgaben_punkte_ges_int)} }
	{ \msg_error:nnn {xsim} {grade-unknown} {#1} }%
}
\cs_new_protected:Npn \xsim_get_points_for_grade_round:n #1
{ \prop_get:NnNTF \l__xsim_relative_grades_prop {#1} \l__xsim_tmpa_tl
	{ \fp_to_decimal:n {round(\l__xsim_tmpa_tl*\schule_aufgaben_punkte_ges_int,0)} }
	{ \msg_error:nnn {xsim} {grade-unknown} {#1} }%
}

% Punkteausgabe von Teilaufgaben an Abi NRW anpassen
\ifdefstring{\schule@klausurtyp}{klausur}{% Klausuren
\str_new:N \ngb_teilpunkte_str
\bool_new:N \ngb_teilpunkte_erste_bool
\bool_gset_true:N \ngb_teilpunkte_erste_bool

\RenewDocumentCommand{\teilaufgabe}{o}{
	\IfInsideSolutionTF{\item}{
		\item%
		\IfNoValueF{#1}{\TP{#1}\xspace}
	}
}

\NewDocumentCommand{\TP}{m}{%
\bool_if:NTF \ngb_teilpunkte_erste_bool {%
	\bool_gset_false:N \ngb_teilpunkte_erste_bool%
	\tl_gput_right:Nn \ngb_teilpunkte_str {#1}%
}{%
	\tl_gput_right:Nn \ngb_teilpunkte_str {\ +\ #1}%
}\addtocounter{teilpunkte}{#1}}

% Teilpunkte zurücksetzen
\newcommand{\TeilpunkteNeu}{%
\str_gclear:N \ngb_teilpunkte_str
\bool_gset_true:N \ngb_teilpunkte_erste_bool
}

% Teilpunkte setzen im Format (a + b + c + ...)
\renewcommand{\Teilpunkte}{\begin{flushright}(\ngb_teilpunkte_str)\end{flushright}}

% Zu Beginn einer Aufgabe Teilpunkte immer auf Null setzen
%\let\oldaufg\aufgabe
%\def\aufgabe{\oldaufg\TeilpunkteNeu}
}{}

\newcommand{\setzeName}[1]{\SetExerciseProperty{ID}{#1}}

\newcommand{\ordnungspunkte}[1]{\begin{aufgabe}[print=false,use=true,ID=OP,points=#1]
	% Empty
\end{aufgabe}}

\str_new:N \ngb_kompetenzen_str
\bool_new:N \ngb_kompetenzen_setzen_bool
\bool_new:N \ngb_erwartungen_erste_bool

\NewDocumentCommand{\ngb@erwartung}{m m O{}}{%
	\bool_if:NTF \ngb_erwartungen_erste_bool {
		\bool_gset_false:N \ngb_erwartungen_erste_bool
	}{
		\tl_gput_right:Nn \schule_zeilen_erwartungen_str {\detokenize{\newline}}
	}

	\int_gadd:Nn \schule_aufgaben_punkte_int {\_str_to_int_with_zero:n{#2}} %Punkte
	\int_gadd:Nn \schule_aufgaben_zusatzpunkte_int {\_str_to_int_with_zero:n{#3}} %Zusatzpunkte
%	\tl_gput_right:Nn \schule_zeilen_erwartungen_str {\detokenize{\newline}}
	\tl_gput_right:Nn \schule_zeilen_erwartungen_str {\detokenize{\dots\; #1}}
	\iftoggle{ngb@teilpunkteAnzeigen}{
		\tl_gput_right:Nn \schule_zeilen_erwartungen_str {\detokenize{\ }(\_ngb_punkte_anzeige:nn {\_str_to_int_with_zero:n{#2}}{\_str_to_int_with_zero:n{#3}})}
	}{}
}

\cs_new:Npn \_ngb_punkte_anzeige:nn #1 #2 {
	\int_compare:nT {#1 > 0}{#1}
	\int_compare:nT {#2 > 0}{\space (+#2)}
}

% Zeile für die Aufgabe in der Tabelle setzen (Standard).
\cs_new:Npn \_ngb_aufgaben_erwartungen_zeile {
	\detokenize{\bfseries }%
	\ifstrequal{\GetExerciseProperty{ID}}{\GetExerciseProperty{id}}{\GetExerciseProperty{counter}}{\GetExerciseProperty{ID}}
	\detokenize{& }%
	\schule_zeilen_erwartungen_str%
	\detokenize{& \bfseries}%
	\_ngb_punkte_anzeige:nn {\int_use:N \schule_aufgaben_punkte_int}{\int_use:N \schule_aufgaben_zusatzpunkte_int}
	\detokenize{& \\}%
	\detokenize{\hline}
}

\NewEnviron{kompetenzen}{
%	\str_gclear:N \ngb_kompetenzen_str
%	\bool_gset_false:N \ngb_kompetenzen_setzen_bool
	\exp_args:Nno\relax{}{\BODY}
}

% Kompetenzen in Form bringen
\NewDocumentCommand{\kompetenz}{m}{%
	\bool_gset_true:N \ngb_kompetenzen_setzen_bool
	\tl_gput_right:Nn \ngb_kompetenzen_str {\detokenize{#1 & }}
	\tl_gput_right:Nn \ngb_kompetenzen_str {\detokenize{ & \\ \hline}}
}

\NewDocumentCommand \GetRelGrade {m}
{ \xsim_get_relative_grade:n {#1} }

\cs_new_protected:Npn \xsim_get_relative_grade:n #1
{
	\prop_get:NnNTF \l__xsim_relative_grades_prop {#1} \l__xsim_tmpa_tl
	{
		\fp_to_decimal:n {round( 100*(\l__xsim_tmpa_tl),0)}\,\%
	}
	{ \msg_error:nnn {xsim} {grade-unknown} {#1} }
}

% Erwartungshorizont (Eine Tabelle für alles)
% --------------------------------------------------------------------
\def\tabularxcolumn#1{m{#1}}
\newcolumntype{M}[1]{>{\large\centering\arraybackslash}m{#1}}
%\newwrite\ehfile

\newcommand{\ngb@erwartungshorizont}{
	%Aufgabenausgabe leeren
	\str_gclear:N \schule_aufgaben_erwartungen_str
	\int_gzero:N \schule_aufgaben_punkte_ges_int
	\int_gzero:N \schule_aufgaben_zusatzpunkte_ges_int
	
	\ForEachUsedExerciseByID{%
		%Variablen für neue Aufgabe neu initialisieren
		\str_gclear:N \schule_zeilen_erwartungen_str
		\int_gzero:N \schule_aufgaben_punkte_int
		\int_gzero:N \schule_aufgaben_zusatzpunkte_int
		\bool_gset_false:N \schule_erwartungen_zeile_gerade_bool
		\bool_gset_true:N \ngb_erwartungen_erste_bool
		
%		\int_gset:Nn \schule_aufgaben_zusatzpunkte_int {\GetExerciseProperty{bonus-points}}
		
		%Definition der Aufgabe in entsprechende Befehle laden
		\def\ExerciseType{##1}%
		\def\ExerciseID{##2}%
%		\GetExercisePropertyTF{erwartungen}{\PropertyValue}{\int_gadd:Nn \schule_aufgaben_punkte_int {\_str_to_int_with_zero:n{\GetExerciseProperty{points}}}\int_gadd:Nn \schule_aufgaben_zusatzpunkte_int {\_str_to_int_with_zero:n{\GetExerciseProperty{bonus-points}}}}
		\GetExercisePropertyTF{erwartungen}{\PropertyValue}{\int_gadd:Nn \schule_aufgaben_punkte_int {\_str_to_int_with_zero:n{\GetExerciseProperty{points}}}}
		
		%Gesamtaufgabe
		\tl_gput_right:Nx \schule_aufgaben_erwartungen_str \_ngb_aufgaben_erwartungen_zeile
		
		\int_gadd:Nn \schule_aufgaben_punkte_ges_int {\schule_aufgaben_punkte_int} %Punkte
		\int_gadd:Nn \schule_aufgaben_zusatzpunkte_ges_int {\schule_aufgaben_zusatzpunkte_int} %Zusatzpunkte
		
		\_schule_erwartungen_punkte_speichern
	}
	
	%Zusammengebautetes wieder serialisieren
	\tokenize{%
		\schule@aEHCode%
	}{%
		\schule_aufgaben_erwartungen_str
	}
	\tokenize{%
		\ngb@aKECode%
	}{%
		\ngb_kompetenzen_str
	}

	% Aux-Datei öffnen
%	\immediate\openout\ehfile=\jobname.eh.aux

%	\immediate\write\ehfile{\unexpanded{Name: \underline{\Large\hspace{6cm}}}}
	\ihead{\small\Fach \Lerngruppe} % Variante auf Horizont aus Kopfzeile nehmen
	\underline{Name: \Large\hspace{6cm}}
	
	\begin{center}
		\renewcommand{\arraystretch}{1.4}
		\begin{tabularx}{\textwidth}{|M{1cm}|X|M{1.5cm}|M{1.5cm}|} \hline
			% Kopfzeile
			\rowcolor{black!20}
			Afg & Die~Schülerin/der~Schüler\dots & mögl.\newline Punkte & \small erreicht \tabularnewline\hline\hline %
			%Erwartungen
			\schule@aEHCode%
			
			% Fusszeile
			\multicolumn{2}{|r|}{Summe:} & \_ngb_punkte_anzeige:nn {\int_use:N \schule_aufgaben_punkte_ges_int}{\int_use:N \schule_aufgaben_zusatzpunkte_ges_int} & \tabularnewline\hline
			%\multicolumn{2}{|r|}{Prozentual:} &  & \tabularnewline\hline
			\multicolumn{2}{|r|}{Note:} & \multicolumn{2}{c|}{} \tabularnewline\hline
		\end{tabularx}
	\end{center}

	\begin{flushright}
		\underline{\today, \hspace{3cm}}
	\end{flushright}
	
	\notenverteilung
	
	\bool_if:NTF \ngb_kompetenzen_setzen_bool {
	\begin{center}
		\renewcommand{\arraystretch}{1.2}
		\begin{tabularx}{\textwidth}{|m{9cm}|X>{\raggedleft\arraybackslash}m{1.5cm}|} \hline
			\rowcolor{black!20}
			Kompetenz & \tiny Kannst~du\newline sehr~gut & \tiny Musst~du~intensiv~üben \\ \hline \hline
			\ngb@aKECode
		\end{tabularx}
	\end{center}
	}{}
	
	% Erwartungshorizont einfuegen
%	\immediate\closeout\ehfile
%	\input{\jobname.eh.aux}
	\clearpage
}

\ifdefstring{\schule@erwartungshorizontStil}{standard}{%
	\RenewDocumentCommand{\erwartung}{m m O{}}{ \ngb@erwartung{#1}{#2}[#3]}%
	\renewcommand{\erwartungshorizont}{\ngb@erwartungshorizont}%
}{}

\ExplSyntaxOff
