\IfEqCase{\schule@klausurtyp}{
	% Klausuren
	{klausur}{%
		% https://tex.stackexchange.com/questions/223694/how-to-draw-a-text-box-with-shadow-borders-i-have-tried-the-following-but-it-gi#223738
		\newcommand{\KlausurTitel}{%
			\begin{tcolorbox}[enhanced,center upper,%
				fontupper=\rmfamily\bfseries,colback=white,%
				drop shadow southeast,sharp corners]
				\Large\Nummer. \TypBezeichnung (\Dauer\,Minuten)\\
				\large\Fach\ \Lerngruppe\ (\Kuerzel)
			\end{tcolorbox}}
		\chead{} % Keine Titel in der Kopfzeile
	}%
}[%
	% Klassen- und Kursarbeiten
	\newcommand{\KlausurTitel}{\begin{center}\LARGE\rmfamily\Titel\end{center}}
	\def\ArbeitTitel{\KlausurTitel}
	
	\ihead{\small\Fach \Lerngruppe\ifthenelse{\equal{\ngb@variante}{}}{}{- \Variante} (\Kuerzel)}
	\chead{\small\Datum}
	%\ohead{\DokumentTyp Nr. \DokumentNummer\linebreak\Datum}
	\ohead{\small\Namensfeld}
	
	\KOMAoptions{headsepline=off}
]

\newcommand{\Dauer}{\makeatletter\ngb@dauer\makeatother\xspace}
\newcommand{\Variante}{\makeatletter\ngb@variante\makeatother\xspace}

%% Anhang in Klausuren/Arbeiten
\RequirePackage{prettyref}
\newrefformat{anhang}{Anhang\,\ref{#1}}
\newcommand\Anhang{\clearpage\appendix\chead{\centering Anhang}}

%% Überschreiben von aufgabeLaden, um die Variante
%% einzubeziehen. Statt <name>.afg.tex wird zuerst nach
%% <name>-<variante>.afg.tex gesucht.
\renewcommand{\aufgabeLaden}[2][\ngb@variante]{%
	\InputIfFileExists{\ngb@afgPool/#2-#1.afg.tex}{}{%
		\InputIfFileExists{\ngb@afgPool/#2.afg.tex}{}{}%
}}


%%%%%%%%%%%%%%%%%%%%%%%%%%%%%%%%%%%%%%%
% Eigene Variante der Notenverteilung %
%%%%%%%%%%%%%%%%%%%%%%%%%%%%%%%%%%%%%%%
\IfEqCase{\ngb@notenStil}{
	{kompakt}{\renewcommand{\notenverteilung}{\Bewertungsschema[]}}
	{ngbschule}{\renewcommand{\notenverteilung}{\Bewertungsschema}}%
	{schule}{}%
}[]

\newcommand{\Bewertungsschema}[1][tendenzen]{
	% Wenn die xsim Umgebung nicht genutzt wurd werden die gesetzten
	% Punkte jetzt übertragen
	\IfExerciseGoalsSumTF{points}{=0}
	{\AddtoExerciseTypeGoal{aufgabe}{ngbpoints}{24}}
	{}
	\ifthenelse{\boolean{schule@kmkPunkte}}{
		\begin{center}\tiny\begin{tabular}{|l||c|c|c|c|c|c|c|c|c|c|c|c|c|c|c|c|} \hline
				\rowcolor{black!20}
				\textbf{Notenpunkte} & 15 & 14 & 13 & 12 & 11 & 10 & 9 & 8 & 7 & 6 & 5 & 4 & 3 & 2 & 1 & 0 \\ \hline
				\textbf{Schwelle} & 95\% & 90\% & 85\% & 80\% & 75\% & 70\% & 65\% & 60\% & 55\% & 50\% & 45\% & 39\% & 33\% & 27\% & 20\% & 0\% \\ \hline
				\rowcolor{black!10}
				\textbf{Punkte}
				& \schule@punkteZuNote{15} 
				& \schule@punkteZuNote{14} 
				& \schule@punkteZuNote{13} 
				& \schule@punkteZuNote{12} 
				& \schule@punkteZuNote{11} 
				& \schule@punkteZuNote{10} 
				& \schule@punkteZuNote{9} 
				& \schule@punkteZuNote{8} 
				& \schule@punkteZuNote{7} 
				& \schule@punkteZuNote{6} 
				& \schule@punkteZuNote{5} 
				& \schule@punkteZuNote{4} 
				& \schule@punkteZuNote{3} 
				& \schule@punkteZuNote{2} 
				& \schule@punkteZuNote{1} 
				& 0 \\ \hline
		\end{tabular}\end{center}
	}{
		\ifthenelse{\equal{#1}{tendenzen}}{
			\begin{center}\renewcommand{\arraystretch}{1.2}\tiny\begin{tabular}{|l||c|c|c|c|c|c|c|c|c|c|c|c|c|c|c|c|} \hline
					\rowcolor{black!20}
					\textbf{Note} & 1+ & 1 & 1- & 2+ & 2 & 2- & 3+ & 3 & 3- & 4+ & 4 & 4- & 5+ & 5 & 5- & 6 \\ \hline
					\textbf{Schwelle} & 95\% & 90\% & 85\% & 80\% & 75\% & 70\% & 65\% & 60\% & 55\% & 50\% & 45\% & 39\% & 33\% & 27\% & 20\% & 0\% \\ \hline
					\rowcolor{black!10}
					\textbf{Punkte}
					& \schule@punkteZuNote{15} 
					& \schule@punkteZuNote{14} 
					& \schule@punkteZuNote{13} 
					& \schule@punkteZuNote{12} 
					& \schule@punkteZuNote{11} 
					& \schule@punkteZuNote{10} 
					& \schule@punkteZuNote{9} 
					& \schule@punkteZuNote{8} 
					& \schule@punkteZuNote{7} 
					& \schule@punkteZuNote{6} 
					& \schule@punkteZuNote{5} 
					& \schule@punkteZuNote{4} 
					& \schule@punkteZuNote{3} 
					& \schule@punkteZuNote{2} 
					& \schule@punkteZuNote{1} 
					& 0 \\ \hline
			\end{tabular}\end{center}
		}{
			\begin{center}\renewcommand{\arraystretch}{1.1}\small\begin{tabular}{|l||c|c|c|c|c|c|} \hline
					\rowcolor{black!20}
					\textbf{Note} & 1 & 2 & 3 & 4 & 5 & 6 \\ \hline
					\textbf{Schwelle} & 85\% & 70\% & 55\% & 39\% & 20\% & 0\% \\ \hline
					\rowcolor{black!10}
					\textbf{Punkte}
					& \schule@punkteZuNote{13}
					& \schule@punkteZuNote{10}
					& \schule@punkteZuNote{7}
					& \schule@punkteZuNote{4}
					& \schule@punkteZuNote{1}
					& 0 \\ \hline
			\end{tabular}\end{center}
		}
	}
}

