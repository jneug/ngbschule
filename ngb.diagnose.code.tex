
\dokumententypName{Diagnose}

\newcommand{\DiagnoseTitel}{{\rmfamily\Large Diagnosebogen \Titel}}

\newcommand{\ngb@personA}{A}
\newcommand{\ngb@personB}{B}

\ihead{\Fach \Lerngruppe (\Kuerzel)\linebreak Diagnosebogen \Titel}
%\chead{\textcolor{ReiheVg}{\Reihe}}
\chead{}
\ohead{\ifthispageodd{Tester: Person \ngb@personA\linebreak Proband: Person \ngb@personB}{Tester: Person \ngb@personB\linebreak Proband: Person \ngb@personA}}
\cfoot{\pagemark}

\RequirePackage{longtable}
\renewcommand{\arraystretch}{1.3}

\newcommand{\smileys}{\Large\usym{1F604}\xspace\usym{1F642}\xspace\usym{1F610}\xspace\usym{1F641}}

\newcounter{ngbDiagItems}

\newenvironment{diagnose}{\begin{center}\begin{tabular}{|p{13cm}|c|} \hline
}{\end{tabular}\end{center}}
\newenvironment{zelle}{}{&}

\newcommand{\diagitem}{\stepcounter{ngbDiagItems}\arabic{ngbDiagItems}.\xspace}