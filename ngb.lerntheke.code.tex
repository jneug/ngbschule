
\RequirePackage[
	margin=1cm,
	includeall,
	headsep=5mm,
	headheight=1.5cm,
	footskip=1.1em,
	footnotesep=0pt,
	marginparwidth=7mm,
	marginparsep=-3mm,
]{geometry}

\RequirePackage[automark,headwidth=\textwidth+1cm]{scrlayer-scrpage}
\RequirePackage{scrhack}

\pagestyle{scrheadings}

%\RequirePackage[usenames,dvipsnames]{xcolor}
\RequirePackage{tikz}
\usetikzlibrary{shapes,shapes.misc,positioning}

\RequirePackage{marginnote}
\RequirePackage{tcolorbox}
\tcbuselibrary{breakable}

\newrefformat{karte}{Station\,\ref{#1}}
\newrefformat{hilfe}{Hilfekarte\,\ref{#1}}
\newrefformat{lsg}{Lösung zu Station\,\ref{#1}}

%\RequirePackage[colorlinks=true,linkcolor=black]{hyperref}

%\usepackage{pgfmorepages}
%\pgfmorepagesloadextralayouts
%\pgfpagesuselayout{4 on 2, odd then even}[a4paper]

\tikzset{
	header/.style={ % requires library shapes.misc
		draw,
		ultra thick,
		chamfered rectangle,
		chamfered rectangle angle=30,
		chamfered rectangle xsep=2cm,
		chamfered rectangle ysep=.38cm,
		minimum width=\textwidth-4cm,
		minimum height=1.5cm,
		font=\sffamily\bfseries\large,
	},
	type1/.style={
		shading=axis,
		bottom color=green!70!black,
		top color=green!40!black,
		shading angle=180,
	},
	type2/.style={
		shading=axis,
		bottom color=yellow!90!black,
		top color=yellow!70!black,
		shading angle=180,
	},
	type3/.style={
		shading=axis,
		bottom color=red!70!gray,
		top color=red!40!black,
		shading angle=180,
	},
	side/.style={
		draw,
		ultra thick,
		fill=gray!10,
		signal,
		signal pointer angle=120,
		minimum width=2.5cm,
		minimum height=1.5cm,
		font=\sffamily\bfseries\large,
		shading=axis,
		bottom color=gray!40,
		top color=gray!80,
		shading angle=180,
	},
	left side/.style={
		side,
		signal from=east,
		signal to=west,
	},
	right side/.style={
		side,
		signal from=west,
		signal to=east,
	},
	side label/.style={
		font=\sffamily\bfseries\Large,
	},
	side label right/.style={
		font=\sffamily\bfseries\Huge,
	},
	side label left/.style={
		side label
	},
	solution/.style={
		fill=yellow!30!white,
		minimum width=\textwidth,
	},
	help/.style={
		shading=axis,
		bottom color=violet!80!white,
		top color=violet!40!black,
		shading angle=180,
		text=white,
	},
	help mark/.style={
		draw,
		black,
		rectangle,
		thick,
		fill=violet,
		text=white,
		minimum size=9mm,
		font=\sffamily\bfseries\large,
		rounded corners=2mm,
	},
}


%%%%%%%%%%%%%%%%%%%%%%%%%%%%%%%%
%         Karteikarten         %
%%%%%%%%%%%%%%%%%%%%%%%%%%%%%%%%
\newcounter{karte}
\newcommand{\karteKopfLinks}{\refstepcounter{karte}%
	\tikz[]{\node[left side,label={[side label left]center:\thekarte}] (left) {};}%
}
\def\Kartennummer{\thekarte}

\def\Kartentitel{}
\def\Kartenart{type1}
\newcommand{\karteKopfMitte}{\addcontentsline{toc}{section}{\Kartentitel}%
\tikz{\node[header,\Kartenart] (header) {\Kartentitel};}%
}

\def\Kartensymbol{}
\newcommand{\karteKopfRechts}[1]{%
\tikz[]{\node[right side,label={[side label right]center:\Kartensymbol}] (right) {};}%
}

\newcommand{\karteKopf}[3][]{\def\Kartensymbol{#1}\def\Kartenart{type#2}\def\Kartentitel{#3}%
\ihead*{\karteKopfLinks}\chead*{\karteKopfMitte}\ohead*{\karteKopfRechts}%
}

\newenvironment{karte}[3][]{%
\karteKopf[#1]{#2}{#3}%
\begin{tcolorbox}[colframe=black,colback=white,rounded corners,width=\textwidth,height=\textheight,left=2mm,right=2mm]}{\end{tcolorbox}\clearpage}

\newenvironment{karte1}[2][]{%
\begin{karte}[#1]{1}{#2}}{\end{karte}}
\newenvironment{karte2}[2][]{%
\begin{karte}[#1]{2}{#2}}{\end{karte}}
\newenvironment{karte3}[2][]{%
\begin{karte}[#1]{3}{#2}}{\end{karte}}

% https://zhiganglu.com/post/latex-insert-blank-page/
\newcommand\leereKarte{
    \null
    \thispagestyle{empty}
    \addtocounter{page}{-1}
    \newpage
}

%%%%%%%%%%%%%%%%%%%%%%%%%%%%%%%%
%        Lösungskarten         %
%%%%%%%%%%%%%%%%%%%%%%%%%%%%%%%%
\newcommand{\loesungKopfMitte}[1]{%
\tikz{\node[header,solution] (header) {\MakeUppercase{#1}};}%
}

\newcommand{\loesungKopf}[1]{%
\ihead*{}\chead*{\loesungKopfMitte{#1}}\ohead*{}%
}
\newenvironment{loesungskarte}[1][Lösung]{%
\loesungKopf{#1}%
\begin{tcolorbox}[colframe=black,colback=white,rounded corners,width=\textwidth,height=\textheight,left=2mm,right=2mm]}{\end{tcolorbox}\clearpage}

%%%%%%%%%%%%%%%%%%%%%%%%%%%%%%%%
%         Hilfekarten          %
%%%%%%%%%%%%%%%%%%%%%%%%%%%%%%%%
\newcounter{hilfekarte}
\renewcommand*{\thehilfekarte}{H\arabic{hilfekarte}}
\newcommand{\hilfekarteKopfLinks}{%
\tikz{\node[left side,label={[side label]center:\thehilfekarte}] (left) {};}%
}

\newcommand{\hilfekarteKopf}[2][H]{\def\Kartensymbol{#1}\def\Kartenart{help}\def\Kartentitel{#2}%
\ihead*{\hilfekarteKopfLinks}\chead*{\karteKopfMitte}\ohead*{\karteKopfRechts}%
}

\newenvironment{hilfekarte}[3][\symHilfe]{%
\refstepcounter{hilfekarte}\hilfekarteKopf[#1]{#2}\label{hilfe:#3}%
\begin{tcolorbox}[colframe=black,colback=white,rounded corners,width=\textwidth,height=\textheight,left=2mm,right=2mm]}{\end{tcolorbox}\clearpage}

%\newcommand{\hilfeMarke}[1]{\marginnote{\begin{tikzpicture}
%\foreach \hm in {#1}
%	\node at (0,{\hm*-8mm}) [help mark] {\hm};
%\end{tikzpicture}}}
\newlength{\markeOffset}\setlength{\markeOffset}{0cm}
\newcommand{\hilfeMarke}[1]{\marginnote{\begin{tikzpicture}
\node at (0,0) [help mark] {\ref{hilfe:#1}};
\end{tikzpicture}}[\markeOffset]\addtolength{\markeOffset}{1.1cm}}


%%%%%%%%%%%%%%%%%%%%%%%%%%%%%%%%
%      Header und Footer       %
%%%%%%%%%%%%%%%%%%%%%%%%%%%%%%%%

\def\@karteFooter{}
\newcommand{\infotext}[1]{\gdef\@karteFooter{#1}\thispagestyle{plain.scrheadings}}
\def\loesung{\infotext}

\clearscrheadings
\addtokomafont{pagefoot}{\scriptsize\color{black!40}}
%\ihead*{\karteKopfLinks}
%\chead*{\karteKopfMitte}
%\ohead*{\karteKopfRechts}
\ifoot*{\Lerngruppe.\Nummer: \Titel, v\Version}
\cfoot*{\lizenzSymbol}
\ofoot[\@karteFooter]{}

%\RequirePackage{rotating}
%\ofoot[\rotatebox{180}{\@karteFooter}]{}