

\newcommand{\ngb@dauer}{}
\newcommand{\ngb@hmft}{0}

\newcommand{\ngb@variante}{}

\newtoggle{ngb@teilpunkteAnzeigen}

\newcommand{\ngb@notenschemaA}{15 = 1,14 = .95,13 = .9,12 = .85,11 = .8,10 = .75,9 = .7,8 = .65,7 = .6,6 = .55,5 = .5,4 = .45,3 = .37,2 = .28,1 = .2}
\newcommand{\ngb@notenschemaB}{15 = .95,14 = .90,13 = .85,12 = .80,11 = .75,10 = .7,9 = .65,8 = .60,7 = .55,6 = .50,5 = .45,4 = .40,3 = .33,2 = .27,1 = .2}

\providebool{schule@kmkPunkte}
\ifdefstring{\schule@typ}{kl}{%
	\ifbool{schule@kmkPunkte}
		{\newcommand{\ngb@notenStil}{schule}}
		{\newcommand{\ngb@notenStil}{ngbschule}}
}{%
	\newcommand{\ngb@notenStil}{ngbschule}
}

\pgfkeys{
	/ngb/.cd,
% Variante und Dauer für Arbeiten/Klausuren
	variante/.store in=\ngb@variante, 
	dauer/.store in=\ngb@dauer,
	hmft/.store in=\ngb@hmft,
	notenverteilungStil/.store in=\ngb@notenStil,
	teilpunkteAnzeigen/.value forbidden,
	teilpunkteAnzeigen/.code=\toggletrue{ngb@teilpunkteAnzeigen},
}