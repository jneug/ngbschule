
%% Kopfzeile anpassen
\dokumententypName{Erwartungshorizont}
\ihead{\Fach \Lerngruppe}
%\chead{Erwartungshorizont \DokumentNummer}
\chead{}
\ohead{\Kuerzel}


\renewcommand{\arraystretch}{1.2}

\newcommand{\HorizontTitel}{\begin{center}\Large\rmfamily\Titel\end{center}}

\newcommand{\Notenfeld}{\begin{flushright}%
		Punkte: \linie[2cm] von \ngb@punkteTotal\\[0,5cm]%
		Notenpunkte: \linie[4cm]\\[0,5cm]%
		\underline{\Heute, \hspace{4cm}}%
	\end{flushright}\par%
}

\newcommand{\Bewertungsschema}[1][tendenzen]{
	% Wenn die xsim Umgebung nicht genutzt wurd werden die gesetzten
	% Punkte jetzt übertragen
	\IfExerciseGoalsSumTF{points}{=0}
		{\AddtoExerciseTypeGoal{aufgabe}{ngbpoints}{24}}
		{}
	\ifthenelse{\boolean{schule@kmkPunkte}}{
		\begin{center}\tiny\begin{tabular}{|l||c|c|c|c|c|c|c|c|c|c|c|c|c|c|c|c|} \hline
			\rowcolor{black!20}
			\textbf{Notenpunkte} & 15 & 14 & 13 & 12 & 11 & 10 & 9 & 8 & 7 & 6 & 5 & 4 & 3 & 2 & 1 & 0 \\ \hline
			\textbf{Schwelle} & 95\% & 90\% & 85\% & 80\% & 75\% & 70\% & 65\% & 60\% & 55\% & 50\% & 45\% & 39\% & 33\% & 27\% & 20\% & 0\% \\ \hline
			\rowcolor{black!10}
			\textbf{Punkte}
			& \schule@punkteZuNote{15} 
			& \schule@punkteZuNote{14} 
			& \schule@punkteZuNote{13} 
			& \schule@punkteZuNote{12} 
			& \schule@punkteZuNote{11} 
			& \schule@punkteZuNote{10} 
			& \schule@punkteZuNote{9} 
			& \schule@punkteZuNote{8} 
			& \schule@punkteZuNote{7} 
			& \schule@punkteZuNote{6} 
			& \schule@punkteZuNote{5} 
			& \schule@punkteZuNote{4} 
			& \schule@punkteZuNote{3} 
			& \schule@punkteZuNote{2} 
			& \schule@punkteZuNote{1} 
			& 0 \\ \hline
		\end{tabular}\end{center}
	}{
		\ifthenelse{\equal{#1}{tendenzen}}{
		\begin{center}\renewcommand{\arraystretch}{1.2}\tiny\begin{tabular}{|l||c|c|c|c|c|c|c|c|c|c|c|c|c|c|c|c|} \hline
				\rowcolor{black!20}
				\textbf{Note} & 1+ & 1 & 1- & 2+ & 2 & 2- & 3+ & 3 & 3- & 4+ & 4 & 4- & 5+ & 5 & 5- & 6 \\ \hline
				\textbf{Schwelle} & 95\% & 90\% & 85\% & 80\% & 75\% & 70\% & 65\% & 60\% & 55\% & 50\% & 45\% & 39\% & 33\% & 27\% & 20\% & 0\% \\ \hline
				\rowcolor{black!10}
				\textbf{Punkte}
				& \schule@punkteZuNote{15} 
				& \schule@punkteZuNote{14} 
				& \schule@punkteZuNote{13} 
				& \schule@punkteZuNote{12} 
				& \schule@punkteZuNote{11} 
				& \schule@punkteZuNote{10} 
				& \schule@punkteZuNote{9} 
				& \schule@punkteZuNote{8} 
				& \schule@punkteZuNote{7} 
				& \schule@punkteZuNote{6} 
				& \schule@punkteZuNote{5} 
				& \schule@punkteZuNote{4} 
				& \schule@punkteZuNote{3} 
				& \schule@punkteZuNote{2} 
				& \schule@punkteZuNote{1} 
				& 0 \\ \hline
		\end{tabular}\end{center}
		}{
		\begin{center}\renewcommand{\arraystretch}{1.1}\small\begin{tabular}{|l||c|c|c|c|c|c|} \hline
				\rowcolor{black!20}
				\textbf{Note} & 1 & 2 & 3 & 4 & 5 & 6 \\ \hline
				\textbf{Schwelle} & 85\% & 70\% & 55\% & 39\% & 20\% & 0\% \\ \hline
				\rowcolor{black!10}
				\textbf{Punkte}
				& \schule@punkteZuNote{13}
				& \schule@punkteZuNote{10}
				& \schule@punkteZuNote{7}
				& \schule@punkteZuNote{4}
				& \schule@punkteZuNote{1}
				& 0 \\ \hline
		\end{tabular}\end{center}
		}
	}
}

%% Makros und Umgebungen für Horizonte ohne das xsim/schule Paket
\newcounter{ngbPunkteTotal}

\newcommand{\ngb@punkteTotal}[1][]{%
	\IfExerciseGoalsSumTF{points}{=0}
		{\ExerciseGoalValuePrint{\value{ngbPunkteTotal}}{}{}}
		{\TotalExerciseGoal{points}{}{}}%
}

\newcommand{\ngb@punkte}[2][0]{#2 \ifnum#1>0 \space(+#1)\fi}
\newcommand{\ngb@aufgabe}[3][0]{\subsection*{Aufgabe #2 (Punkte: \ngb@punkte[#1]{#3})}}

\newcommand{\ngb@horafg}{}
\newcommand{\ngb@horafgBonus}{}

\newenvironment{horizont}[3][0]{%
	\renewcommand{\ngb@horafg}{ngbHorizont-\detokenize{#2}}\newcounter{\ngb@horafg}%
	\renewcommand{\ngb@horafgBonus}{ngbHorizontBonus-\detokenize{#2}}\newcounter{\ngb@horafgBonus}%
	\ngb@aufgabe[#1]{#2}{#3}\vspace*{-2ex}
	\begin{longtable}{|p{.65\linewidth}|c|c|} \hline
		\rowcolor{black!20}
		\textbf{Erwartung} & \textbf{Mögl. Punkte} & \textbf{Punkte} \\ 
		\hline\hline\endhead%
}{%
		\hline\rowcolor{black!10}
		\textbf{Gesamt:} & \textbf{\ngb@punkte[\arabic{\ngb@horafgBonus}]{\arabic{\ngb@horafg}}}%
			\addtocounter{ngbPunkteTotal}{\value{\ngb@horafg}} & \\ \hline
	\end{longtable}%
}

\newcommand{\erwart}[3][0]{\addtocounter{\ngb@horafg}{#3}\addtocounter{\ngb@horafgBonus}{#1}%
	#2 & \textsf{\ngb@punkte[#1]{#3}} & \\ \hline}

\newcommand{\teiler}[1]{\multicolumn{3}{||l||}{\color{gray}\sffamily\bfseries #1} \\ \hline}