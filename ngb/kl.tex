
\IfEqCase{\schule@klausurtyp}{
	% Klausuren
	{klausur}{
		% https://tex.stackexchange.com/questions/223694/how-to-draw-a-text-box-with-shadow-borders-i-have-tried-the-following-but-it-gi#223738
		\newcommand{\KlausurTitel}{%
			\begin{tcolorbox}[enhanced,center upper,%
				fontupper=\rmfamily\bfseries,colback=white,%
				drop shadow southeast,sharp corners]
				\Large\DokumentNummer. Klausur (\Dauer Minuten)\\
				\large\Fach\ \Lerngruppe\ (\Kuerzel)
			\end{tcolorbox}}
		
		\chead{} % Keine Titel in der Kopfzeile
	}%
}[%
	% Klassen- und Kursarbeiten
	\newcommand{\KlausurTitel}{\begin{center}\LARGE\rmfamily\Titel\end{center}}
	\def\ArbeitTitel{\KlausurTitel}
	
	\ihead{\small\Fach \Lerngruppe\ifthenelse{\equal{\ngb@variante}{}}{}{- \Variante} (\Kuerzel)}
	\chead{\small\Datum}
	%\ohead{\DokumentTyp Nr. \DokumentNummer\linebreak\Datum}
	\ohead{\small\Namensfeld}
	
	\KOMAoptions{headsepline=off}
]

\newcommand{\Dauer}{\makeatletter\ngb@dauer\makeatother\xspace}
\newcommand{\Variante}{\makeatletter\ngb@variante\makeatother\xspace}

%% Anhang in Klausuren/Arbeiten
\RequirePackage{prettyref}
\newrefformat{anhang}{Anhang\,\ref{#1}}
\newcommand\Anhang{\clearpage\appendix\chead{\centering Anhang}}

\renewcommand{\aufgabeLaden}[2][\ngb@variante]{%
	\InputIfFileExists{\ngb@afgPool/#2-#1.tex}{}{%
		\InputIfFileExists{\ngb@afgPool/#2.tex}{}{}%
}}