
%% -----------------
%%
%% Makros für Dokumente im Fach Informatik
%% Vor allem Eindeutschung und Kuerzel für Struktogramm und UML Makros
%%
%% -----------------

%% Styling von Quelltexten
\lstset{
	basicstyle=\small\ttfamily,
	keywordstyle=\color{black}\bfseries,
	identifierstyle=, % nothing happens
	commentstyle=\color{gray},
	stringstyle=,
	showstringspaces=false,
	numbers=left,
}

\def\zeilennummernAus{\lstset{numbers=none}}
\def\zeilennummernEin{\lstset{numbers=left}}
\def\setzeSprache#1{\lstset{language=#1}}

% Struktogramme (struktex)
\sBoolValue{\texttt{true}}{\texttt{false}}
\newcommand{\wenndann}[4][8]{\ifthenelse[#1]{#2}{#3}{#4}{\sTrue}{\sFalse}}
\newcommand{\anweisung}[2][8]{\assign[#1]{#2}}
\newcommand{\leer}[1][8]{\assign[#1]{}}
\def\sonst{\change}
\def\wennende{\ifend}

% UML (pgf-umlcd)
\def\attribut{\attribute}
\def\methode{\operation}
\newenvironment{klasse}[3][text]{\begin{class}[#1]{#2}{#3}}{\end{class}}
\def\benutzt{\unidirectionalAssociation}
% Selbstassoziation
% Von: https://tex.stackexchange.com/questions/98021/how-to-extend-pgf-umlcd-with-self-association-connection#98023
\newcommand{\benutztSelbst}[3]{%
\coordinate (a) at ($(#1.east) + (0,1)$);
\coordinate (b) at ($(#1.east) + (1,1)$);
\coordinate (d) at ($(#1.east) + (1,-1)$);
\coordinate (e) at ($(#1.east) + (0,-1)$);
\coordinate (t) at ($(#1.east) + (1,0)$);
\coordinate (c) at ($(d)!(b)!(t)$);
  \draw [umlcd style,<-] (a) -- (b)
  node[midway, above]{#2}
  node[midway, below]{#3};
  \draw [umlcd style] (b) -- (c);
  \draw [umlcd style] (c) -- (d);
  \draw [umlcd style] (d) -- (e);
}

%% Laden von vordefinierten Struktogrammen, 
%% Klassendiagrammen und Quelltexten aus der internen
%% Bibliothek
\newcommand{\nss}[1]{\centernssfile{if/#1}}
\newcommand{\cd}[1]{\input{if/#1.cd.tex}}
\newcommand{\lst}[1]{\input{if/#1.lst.tex}}