% !TeX root = ngbschule-doc
\section{Klausuren und Klassenarbeiten}

\begin{options}
	\opt{variante}
	Variante der Klausur/Klassenarbeit. Sinnvoll, wenn A/B Arbeiten oder generell verschiedene Versionen gesetzt werden sollen. Kann im Dokument über \cs*{Variante} abgerufen werden.
	
	\opt{dauer}
	Dauer der Klausur/Klassenarbeit. Kann im Dokument über \cs*{Dauer} abgerufen werden.
	
	\keychoice{notenverteilungStil}{schule,ngbschule,kompakt}
	Setzt den Stil der Notenverteilungstabelle.
	
	\opt{teilpunkteAnzeigen}
	Wird ein Erwartungshorizont angezeigt und der Stil nicht explizit auf \texttt{schule} gesetzt, kann mit dieser Option hinter jeder Erwartung die erreichbare Punktzahl angezeigt werden. Im Normalfall wird im neuen Erwartungshorizont nur die Gesamtzahl der Punkte angezeigt.
\end{options}

\begin{commands}
	\command{tier}
	Fügt ein zufällig gewähltes Tierbild ein, dass unterhalb von Klassenarbeiten als Motivation eingesetzt werden kann.
	
	\command{vielErfolg}
	Fügt den Text \enquote{Viel Erfolg!} und ein zufällig gewähltes Tierbild mit \cs*{tier} ein.
	
	\command{Bewertungsschema}[\oarg{t}]\Default{tendenzen}
\end{commands}

\subsection{Erwartungshorizonte}