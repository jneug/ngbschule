\PassOptionsToPackage{colaction}{multicol}
\documentclass[a4paper,add-index,ngerman]{cnltx-doc}

\usepackage[utf8]{inputenc}
\usepackage[ngerman]{babel}

\usepackage{tcolorbox}

\usepackage{booktabs}
\usepackage{blindtext}
\usepackage{weva}
\usepackage{todonotes}
\usepackage{pdfpages}

\usepackage[
	typ=ohne,
	fach=ohne,
	weitereFaecher={
		Informatik,
		Mathematik
	},
	farbig,
	module={}
]{schule}

\usepackage[
	kuerzel={Ngb},
	reihe={Beispiel einer Reihe},
	keineMetadaten,
]{ngbschule}

\title{Dokumentation des Pakets ngbschule}

\setcnltx{
	name     = ngbschule ,
	title    = ngbschule ,
	version  = 0.0.6 ,
	date     = 2018-12-15 ,
	subtitle = {\LaTeX-Paket zur Erweiterung des schule-Pakets.},
	info     = Paketdokumentation ,
	authors  = {Jonas Neugebauer} ,
	email    = schule@neugebauer.cc ,
	url     = https://ngb.schule ,
	abstract = {%
		Erweiterungen und Ergänzungen des schule-Pakets.
	} ,
	index-setup = { othercode=\footnotesize,level=\section},
%	add-listings-options= {
%		morekeywords={
%			
%		}
%	},
}

\newnote*\sidenote[1]{#1}

\usepackage{prettyref}
\newrefformat{part}{Kapitel\,\ref{#1}, S.\,\pageref{#1}}
\newrefformat{sec}{Abschnitt\,\ref{#1}, S.\,\pageref{#1}}

%\includeonly{allgemeines, faecher}

\begin{document}

\part{Einleitung}\label{part:einleitung}

\section{Typografie und Farben}\label{sec:typografie}
Die Gestaltung der Dokumente wird zum größten Teil in der Datei \texttt{theme.tex} festgelegt, die per \cs*{input} eingebunden wird.

Derzeit wird für alle Dokumente dieselbe Theme-Datei eingebunden, allerdings ist eine Idee in Zukunft unterschiedliche Themes für Material der Sekundarstufe I und II einzubauen.

\subsection{Schriftarten}\label{sec:schriften}
Als Schriftarten werden \pkg{FiraSans} für Textabsätze und \pkg{tgschola} für Überschriften genutzt. Als Monospace-Font wird \pkg{courier} geladen.

\subsubsection{Überschriften}\label{sec:ueberschriften}
Generell werden die Kommandos zum ändern von Schriftarten des \texttt{KOMA}-Paketes genutzt (\cs*{setkomafont} und \cs*{addtokomafont}), um die Schriftarten anzupassen.

Zusätzlich zu den dort verfügbaren gibt es eine neue Überschrift für den Reihentitel (\texttt{reihe}). Wie alle \texttt{KOMA} Schriften kann sie über \cs*{usekomafont}\Marg{reihe} genutzt werden, was  aber in der Regel von den entsprechenden Makros für Reihe und Titel automatisch übernommen wird.

\subsection{Farben}\label{sec:farben}

\part{Präambel}
\section{Pakete einbinden}
\section{Dokumentinformationen}

\part{Allgemeine Erweiterungen}\label{part:allgemeines}

\begin{options}
	\opt{ersteAufgabe}\Default{1} Setzt die Nummer der ersten Aufgabe im Dokument.
\end{options}

\begin{commands}
	\command{linie}[\oarg{breite}]\Default{6cm}
	Zeichnet eine Linie der angegebenen Breite.
	\begin{example}
	\linie[4cm]
	\end{example}

	\command{Namensfeld}[\oarg{breite}]\Default{5cm}
	Erzeugt ein Feld, um einen Namen einzutragen.
	
	\begin{example}
	\Namensfeld
	\end{example}
	
	\command{tab}
	Kurz für \verbcode!\hspace{1cm}!.

	\command{titlerule}
	Erzeugt eine Linie mit Breite \cs*{textwidth} und Dicke 1pt, die als Unterstreichung von Überschriften benutzt wird..
	
	\command{code}[\marg{text}]
	Wird zur Auszeichnung von Quelltextauszügen benutzt. Das Makro wird auch von anderen Paketen erzeugt, daher prüft \ngbschule\ zunächst, ob es schon existiert und erzeugt nur dann eine eigene Version.

	\command{operator}[\marg{operator}]
	Wird benutzt, um in Aufgabentexten Operatoren auszuzeichnen.
	\begin{example}
	\operator{Implementiere} die Operation \texttt{push} aus der Klasse \texttt{Stack}.
	\end{example}

	\command{TITEL}
	Erzeugt einen großen, zentrierten Titel.
	\begin{example}
	\TITEL
	\end{example}

	\command{ReiheTitel}
	Erzeugt einen Dokumenttitel bestehend aus Reihe und Titel, sowie einem \cs*{titlerule} als Trenner.
	\begin{example}
	\ReiheTitel
	\end{example}
\end{commands}

\begin{environments}
	\environment{rahmen}[\oarg{color}]\Default{black}
	Erzeugt eine \pkg{tcolorbox} mit einem schwarzen einfachen Rahmen.
	
	\begin{example}
	\begin{rahmen}
	\blindtext
	\end{rahmen}
	\end{example}
\end{environments}


\section{Aufzählungen}\label{sec:aufzaehlungen}
Für Aufzählungen werden die Pakete \pkg{enumitem} (von \pkg*{schule}) und \pkg{tasks} eingebunden

Das Paket ändert die von \pkg*{schule} ergänzte Aufzählung \env{smallenumerate} so um, dass als Aufzählungszeichen \code{a)}, \code{b)}, \code{c)}, ... benutzt werden. Dadurch wird sie für Listen Unteraufgaben leichter nutzbar.

Durch Kombination mit der Umgebung \env{multicols} können die normalen Listen auch mehrspaltig gesetzt werden.

\begin{example}
	\begin{multicols}{3}
		\begin{smallenumerate}
			\item Item 1
			\item Item 2
			\item Item 3
			\item Item 4
			\item Item 5
		\end{smallenumerate}
	\end{multicols}
\end{example}

Neben \env*{smallenumerate} gibt es die beiden Umgebungen \env{enumeratea} und \env{enumeraten}, die jeweils alphabetische und numerische Listen erzeugen,

\begin{example}
	\begin{multicols}{3}
		\begin{enumeratea}
			\item Item 1
			\item Item 2
			\item Item 3
		\end{enumeratea}
	\end{multicols}
\end{example}


\begin{example}
	\begin{multicols}{4}
		\begin{enumeraten}
			\item Item 1
			\item Item 2
			\item Item 3
		\end{enumeraten}
	\end{multicols}
\end{example}

Die Umgebung \env*{multicols} setzt den Inhalt in mehreren Spalten. Soll eine \enquote{echte} horizontale Liste erzeugt werden, dann sollte die Umgebung \env{tasks}\Darg{cols} benutzt werden. Statt mit \cs*{item} werden in dieser die einzelnen Elemente mit \cs{task} gesetzt. Die Anzahl der Spalten wird in \emph{runden} Klammern angegeben. 

\begin{example}
	\begin{tasks}(4)
		\task Item 1
		\task Item 2
		\task Item 3
		\task Item 4
		\task Item 5
	\end{tasks}
\end{example}

\section{Aufgabensammlung}\label{sec:aufgabensammlung}
Um das Aufbauen einer Aufgabensammlung zu vereinfachen steht allen Dokumenttypen das Makro \cs*{aufgabeLaden}\marg{name} zur Verfügung. Mit diesem können Aufgabendateien (mit der Endung \texttt{.afg.tex}) aus einem Ordner in das aktuelle Dokument eingefügt werden. Der Pfad, in dem nach den Aufgabendateien gesucht wird kann durch die Option \option*{aufgaben} für jedes Dokument angepasst werden.

\begin{options}
	\keyval{aufgaben}{pfad}\Default{ngb/afg}
	\sidenote{In zukünftigen Versionen sollte der Pfad für die Sammlung zentral anpassbar sein.}
	Setzt den Pfad zur Aufgabensammlung. Als Standard wird in einem Unterordner des Pakets selbst gesucht, von dem aus die Aufgaben jeder \TeX Datei zur Verfügung stehen. Um die Aufgabensammlung an einem anderen Ort zu speicher muss (derzeit) der Pfad in jeder Datei über diese Option angegeben werden.
	
	Der Pfad kann absolut sein (z.B. \texttt{~/user/tex/Aufgaben}) oder eine relative Pfadangabe (z.B. \texttt{../../Aufgaben}).
\end{options}

\begin{commands}
	\command{aufgabeLaden}[\marg{name}]
	Lädt eine Aufgabe aus der Aufgabensammlung, wenn sie vorhanden ist. Die Datei muss den Namen \texttt{\meta{name}.afg.tex} haben und im Ordner der Aufgabensammlung liegen.
	
	Ohne Angabe eines neuen Pfades für die Aufgabensammlung
	ist der zusammengesetzte Pfad für eine Aufgabe dann also
	\texttt{ngb/afg/\meta{name}.afg.tex}.
	
	Es können auch Unterordner angegeben werden. (Z.B. \cs*{aufgabeLaden}\Marg{if/aufgabe-1}. Die
	Aufgabendatei liegt in dem Fall also unter \texttt{ngb/afg/if/aufgabe-1.afg.tex}.)
\end{commands}


\section{Materialsammlung}\label{sec:materialsammlung}
Das Paket liefert eine Reihe von Materialien mit, die (hauptsächlich bezogen auf die Lehrpläne in NRW) häufig in Dokumenten auftauchen. Die einzelnen Fächer stellen Makros bereit, um Fachspezifische Materialien nachzuladen. Dies ist vor allem für die Informatik relevant, für das vorgefertigte \emph{UML-Diagramme}, \emph{Struktogramme} und \emph{Quelltexte} eingebunden werden können.

Details dazu sind bei der Beschreibung der Fächer unter \prettyref{sec:materialinfo} und \prettyref{sec:materialmathe} beschrieben.

Generell sollten die Materialdateien möglichst so gehalten sein, dass sie auch ohne das Paket \ngbschule\ verwendet werden können (Struktogramme sollten also zum Beispiel nicht die Makros \cs*{Anweisung} etc. verwenden, sondern direkt die Makros aus dem \pkg*{stuktex}-Paket).

\part{Dokumenttypen und Subtypen}
\section{Arbeitsblätter}
\subsection{Allgemeine Arbeitsblätter}
\begin{commands}
	\command{qrhinweis}[\oarg{qrtext}\marg{text}]
	Erstellt einen Bearbeitungshinweis in Form eines QR-Codes. Das Paket \pkg{qrcode} muss separat im Dokument nachgeladen werden.
	
	Wird die Option \option{hinweise} angegeben, dann werden statt QR-Codes die Hinweise im Klartext angezeigt.
	
	Wird die \pkg*{schule}-Option \option*{loesungen} auf etwas anderes als \texttt{keine} gesetzt wird automatisch auch \option*{hinweise} gesetzt!
	
	Da die Codes keine mit \TeX\ gesetzten Formatierungen darstellen können, kann durch den optionalen Parameter \meta{qrtext} ein alternativer Text für den QR-Code angegeben werden. Dadurch können für die Lösungen die Hinweise schön formatiert und dennoch als QR-Code gesetzt werden.
\end{commands}

\subsection{Checkup und Selbstlernbögen}
\subsection{Diagnosebögen}

\section{Klausuren und Klassenarbeiten}
\subsection{Klausuren}
\subsection{Erwartungshorizonte}

% !TeX root = ngbschule-doc
\part{Fächer}\label{part:faecher}

\section{Fach Informatik}\label{sec:fachinfo}

\subsection{Quelltexte in Tabellen}
...

\subsection{Materialien Informatik}\label{sec:materialinfo}
\sidenote{Sobald für mich relevant werden auch Sequenzdiagramme und PAPs ergänzt.}
Für das Fach Informatik lassen sich auf einfache Weise eine Reihe vorgefertigte UML-Diagramme, Struktogramme und Quelltexte einbinden.

\begin{commands}
	\command{cd}[\marg{name}]
	Fügt das Klassendiagramm mit dem Namen \meta{name} in das Dokument ein.
	
	\command{lst}[\marg{name}]
	Fügt die Quelltextdatei mit dem Namen \meta{name} in das Dokument ein.
	
	\command{nss}[\marg{name}]
	Fügt das Struktogramm mit dem Namen \meta{name} in das Dokument ein.
\end{commands}

Die Namen der Materialien halten sich an die Punktnotation, um eine möglichst einfache Organisation zu gewährleisten. Die Namen lassen sich so auch mehrfach verwenden, um zum Beispiel verschiedene Darstellungen eines Algorithmus darzustellen.

Die Operation \texttt{pop} des \texttt{Stack} ist zum Beispiel unter dem Namen \texttt{stack.pop} gespeichert. Mit \cs*{lst}\Marg{stack.pop} wird eine Quelltextdarstellung der Operation eingebunden (Standard: Java), während \cs*{nss}\Marg{stack.pop} das entsprechende Struktogramm anzeigt. \cs*{cd}\Marg{stack} bindet wiederum ein Klassendiagramm des \texttt{Stack} ein und \cs*{lst}\Marg{stack.pop.pseudo} einen Pseudocode.

Alle Dateien für das Fach Informatik liegen im Unterordner \texttt{if}.Die Dateinamen setzen sich aus dem Namen des Materials und einer Endung pro Typ zusammen (abgeleitet aus dem Namen des Makros, z.B. \texttt{.cd.tex} für \cs*{cd}).

\subsubsection{UML-Diagramme}
Klassen- und Objektdiagramme liegen in Dateien mit der Endung \texttt{.cd.tex} im Materialordner. \\Zum Beispiel \texttt{ngb/inc/if/stack.cd.tex}. Dabei ist zu beachten, dass die Namen, die Operationen/Methoden entsprechen, in 
der Regel Objektdiagramme enthalten, die die Veränderungen der Zeiger beim Aufruf der Operation darstellen.

\begin{multicols}{2}\ttfamily
	\begin{smallitemize}
		\item stack
		\item stack.pop
		\item stack.push
		\item queue
		\item queue.enqueue
		\item queue.dequeue
		\item list
		\item list.insert
		\item list.append
		\item list.delete
	\end{smallitemize}
\end{multicols}

\subsubsection{Quelltexte}
Quelltexte liegen in Dateien mit der Endung \texttt{.lst.tex} im Materialordner. \\Zum Beispiel \texttt{ngb/inc/if/stack.pop.lst.tex}.
Bei mehreren verfügbaren Sprachen wird der Name der Sprache angehängt, also zum Beispiel \texttt{stack.pop.java.lst.tex}. Das Makro
sucht zunächst nach einer Datei mit dem exakten angegebenen Namen und als Fallback nach der Java-Version.

\begin{multicols}{2}\ttfamily
	\begin{smallitemize}
		\item stack.pop.java
		\item stack.push.java
		\item queue.enqueue.java
		\item queue.dequeue.java
		\item list.insert.java
		\item list.append.java
		\item list.delete.java
		\item stack.pop.pseudo
		\item stack.push.pseudo
		\item queue.enqueue.pseudo
		\item queue.dequeue.pseudo
		\item list.insert.pseudo
		\item list.append.pseudo
		\item list.delete.pseudo
		\item stack.pop.python
		\item stack.push.python
		\item queue.enqueue.python
		\item queue.dequeue.python
		\item list.insert.python
		\item list.append.python
		\item list.delete.python
	\end{smallitemize}
\end{multicols}

\subsubsection{Struktogramme}
Struktogramme liegen in Dateien mit der Endung \texttt{.nss} im Materialordner. \\Zum Beispiel \texttt{ngb/inc/if/stack.pop.nss}.

\begin{multicols}{2}\ttfamily
	\begin{smallitemize}
		\item stack.pop
		\item stack.push
		\item queue.enqueue
		\item queue.dequeue
		\item list.insert
		\item list.append
		\item list.delete
	\end{smallitemize}
\end{multicols}

\section{Fach Mathematik}\label{sec:fachmathe}

Für Dokumente mit dem Typ \enquote{ab} (\keyis*-{typ}{ab}) werden weitere Pakete nachgeladen:
\begin{multicols}{2}
	\begin{smallitemize}
		\item \pkg{amsmath}
		\item \pkg{amssymb}
		\item \pkg{amstext}
		\item \pkg{amscd}
		\item \pkg{exscale}
		\item \pkg{tkz-base}
		\item \pkg{tkz-euclide} und \cs*{usetkzobj}\Marg{all} wird ausgeführt.
		\item \pkg{tkz-fct} benötigt eine Installation von \texttt{gnuplot} und \texttt{pdflatex} muss mit dem Schalter \texttt{--shell-escape} gestartet werden (\zB\  \texttt{pdflatex --shell-escape "ab-mit-tkz.tex"}).
		\item \pkg{skmath} mit den Optionen \option*{commonsets} und \keyis*-{notation}{german}.
		\item \pkg{units}
		\item \pkg{gauss}
	\end{smallitemize}
\end{multicols}

Die Dokumentation der Pakete hilft bei der Verwendung, aber gerade für die Geometrie-Pakete (\pkg*{tkz}) werden viele hilfreiche Kommandos ergänzt (siehe dazu \ref{sec:geometrie}).

\subsection{Mathematische Erweiterungen}\label{sec:allgemeinmathe}
\begin{environments}
	\environment{sachaufgabe}
	Setzt die Lösung einer Sach-/Textaufgabe mit \cs{frage}, \cs{rechnung} und \cs{antwort}. Die Umgebung hat nur semantische Bedeutung, da sie einfach eine \env*{description} Umgebung darstellt.
	
	\begin{example}
		\begin{sachaufgabe}
			\frage Wieviele Enten schwimmen auf dem See?
			\rechnung $\text{Alle Meine Entchen} - \text{Meine Entchen} = \text{Alle}$
			\antwort \emph{Alle} Entchen schwimmen auf dem See.
		\end{sachaufgabe}
	\end{example}
\end{environments}
\begin{commands}
	\command{grad}[\marg{gradzahl}] Formatiert \meta{gradzahl} mit dem Grad-Symbol ($^\circ$).
	
	\command{bs} Erzeugt einen backslash: $\bs$
	
	\command{qed} Erzeugt eine Box als Abschluss von Beweisen: \qed
\end{commands}

\subsection{Geometrische Konstruktionen}\label{sec:geometrie}

\subsection{Materialien Mathematik}\label{sec:materialmathe}
Für Mathematik sind vor allem geometrische Konstruktionen und Merksätze / Definitionen angedacht.

\subsection{Hinweise zum Formelsatz}\label{sec:formelsatz}
Um die Darstellung von Formeln möglichst einheitlich zu gestalten, werden im Folgenden die verschiedenen mathematischen Umgebungen verglichen und Beispiele für wiederkehrende Formeln wie Termumformungen oder Gleichungssysteme gegeben.

\subsubsection*{Mathematische Umgebungen}
\begin{itemize}
	\item equation
	\item array
	\item eqnarray
\end{itemize}

\subsubsection*{Formelsatz}

Für Gleichungssysteme und Termumformungen wird das Paket \pkg{gauss} eingebunden. Um ein Gleichungssysteme zu setzen bietet sich die Umgebung \env{alignedat} an. Diese setzt nicht automatisch Fformeln und muss daher explizit in den Math-Mode gesetzt werden:
\begin{example}
\[ \begin{gmatrix}[v]
	-2x & + 3y & = & 4 \\
	3x & -2y + & 5z = & 9 \\
	x & + y + & 5z = & 13
	\end{gmatrix} \]
\end{example}
	
		Durch Umformungen erhält man:
		\[ \begin{gmatrix}[v]
		-2x  & +3y & & = 4 \\
		3x & -2y & +5z & = 9 \\
		x & +y & +5z&  = 13
		\rowops
		\mult{2}{\cdot 2}\add{2}{0}
		\mult{2}{\cdot -3}\add{2}{1}
		\end{gmatrix} \]
		\[ \begin{gmatrix}[v]
		& +5y & + 10z & = 30 \\
		& -5y & - 10z & = -30 \\
		x & +y & +5z&  = 13
		\rowops
		\add{0}{1}
		\end{gmatrix} \]
		\[ \begin{gmatrix}[v]
		& +5y & + 10z & = 30 \\
		& & 0 & = 0 \\
		x & +y & +5z&  = 13
		\end{gmatrix} \]

\part{Anhang}

\end{document}