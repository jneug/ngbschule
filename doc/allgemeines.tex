% !TeX root = ngbschule-doc
\part{Allgemeine Erweiterungen}\label{part:allgemeines}

\begin{options}
	\opt{ersteAufgabe}\Default{1} Setzt die Nummer der ersten Aufgabe im Dokument.
\end{options}

\begin{commands}
	\command{linie}[\oarg{breite}]\Default{6cm}
	Zeichnet eine Linie der angegebenen Breite.
	\begin{sidebyside}
	\linie[4cm]
	\end{sidebyside}

	\command{Namensfeld}[\oarg{breite}]\Default{5cm}
	Erzeugt ein Feld, um einen Namen einzutragen.
	
	\begin{sidebyside}
	\Namensfeld
	\end{sidebyside}
	
%	\command{tab}
%	Kurz für \verbcode!\hspace{1cm}!.

	\command{titlerule}
	Erzeugt eine Linie mit Breite \cs*{textwidth} und Dicke 1pt, die als Unterstreichung von Überschriften benutzt wird.
	
	\command{code}[\marg{text}]
	Wird zur Auszeichnung von Quelltextauszügen benutzt. Das Makro wird auch von anderen Paketen erzeugt, daher prüft \ngbschule\ zunächst, ob es schon existiert und erzeugt nur dann eine eigene Version.
	\begin{sidebyside}
	\code{int a = 1;}
	\end{sidebyside}
	
	\command{appfunktion}[\marg{text}]
	Wird zur Auszeichnung von Funktionen, Menüs und Tasten in Apps benutzt.
	\begin{sidebyside}
	\appfunktion{Ansicht}
	\end{sidebyside}
	
	
	\command{tipp}
	Auszeichnung für einen Tipp z.B. auf einem Arbeitsblatt. Prinzipiell gleich zu \cs*{hinweis}.

	\command{operator}[\marg{operator}]
	Wird benutzt, um in Aufgabentexten Operatoren auszuzeichnen.
	\begin{sidebyside}
	\operator{Implementiere} die Operation \texttt{push} aus der Klasse \texttt{Stack}.
	\end{sidebyside}

	\command{TITEL}
	Erzeugt einen großen, zentrierten Titel.
	\begin{example}
	\TITEL
	\end{example}

	\command{ReiheTitel}
	Erzeugt einen Dokumenttitel bestehend aus Reihe und Titel, sowie einem \cs*{titlerule} als Trenner.
	\begin{example}
	\ReiheTitel
	\end{example}
\end{commands}

\begin{environments}
	\environment{rahmen}[\oarg{color}]\Default{black}
	Erzeugt eine \pkg{tcolorbox} mit einem schwarzen einfachen Rahmen.
	
	\begin{example}
	\begin{rahmen}
	\blindtext
	\end{rahmen}
	\end{example}
\end{environments}


\section{Aufzählungen}\label{sec:aufzaehlungen}
Für Aufzählungen werden die Pakete \pkg{enumitem} (von \pkg*{schule}) und \pkg{tasks} eingebunden

Das Paket ändert die von \pkg*{schule} ergänzte Aufzählung \env{smallenumerate} so um, dass als Aufzählungszeichen \code{a)}, \code{b)}, \code{c)}, ... benutzt werden. Dadurch wird sie für Listen Unteraufgaben leichter nutzbar.

Durch Kombination mit der Umgebung \env{multicols} können die normalen Listen auch mehrspaltig gesetzt werden.

\begin{example}
	\begin{multicols}{3}
		\begin{smallenumerate}
			\item Item 1
			\item Item 2
			\item Item 3
			\item Item 4
			\item Item 5
		\end{smallenumerate}
	\end{multicols}
\end{example}

Neben \env*{smallenumerate} gibt es die beiden Umgebungen \env{enumeratea} und \env{enumeraten}, die jeweils alphabetische und numerische Listen erzeugen,

\begin{example}
	\begin{multicols}{3}
		\begin{enumeratea}
			\item Item 1
			\item Item 2
			\item Item 3
		\end{enumeratea}
	\end{multicols}
\end{example}


\begin{example}
	\begin{multicols}{3}
		\begin{enumeraten}
			\item Item 1
			\item Item 2
			\item Item 3
		\end{enumeraten}
	\end{multicols}
\end{example}

Die Umgebung \env*{multicols} setzt den Inhalt in mehreren Spalten. Soll eine \enquote{echte} horizontale Liste erzeugt werden, dann sollte die Umgebung \env{tasks}\Darg{cols} benutzt werden. Statt mit \cs*{item} werden in dieser die einzelnen Elemente mit \cs{task} gesetzt. Die Anzahl der Spalten wird in \emph{runden} Klammern angegeben. 

\begin{example}
	\begin{tasks}(4)
		\task Item 1
		\task Item 2
		\task Item 3
		\task Item 4
		\task Item 5
	\end{tasks}
\end{example}

\section{Zahlen und Einheiten}\label{sec:zahlen}
Zahlen im Fließtext und Werte mit Einheiten werden mit dem Paket \pkg{siunitx} gesetzt. Für das Eurosymbol bindet \pkg*{schule} schon das Paket \pkg{eurosym} ein.

\subsection{Zahlen}
Das Kommando \cs{num} formatiert Zahlen automatisch in ein einheitliches Format. Im Text-Modus ist es der bevorzugte Weg, Zahlen darzustellen.

\subsection{Einheiten}
Größen mit Einheiten werden mit den Kommandos \cs{si} und \cs{SI} gesetzt. Erstere Variante setzt nur eine Einheit (z.B. \verb!\si{\cubic\meter}! = \si{\cubic\meter}), während die zweite Variante noch den Wert der Größe bekommt (z.B. \verb!\SI{5}{\square\centi\meter}! = \SI{5}{\square\centi\meter}).

Geldbeträge in Euro können direkt mit dem Kommando \cs{EUR} gesetzt werden: \verb!\EUR{12.2}! = \EUR{12.2}.

\section{Aufgabensammlung}\label{sec:aufgabensammlung}
Um das Aufbauen einer Aufgabensammlung zu vereinfachen steht allen Dokumenttypen das Makro \cs*{aufgabeLaden}\marg{name} zur Verfügung. Mit diesem können Aufgabendateien (mit der Endung \texttt{.afg.tex}) aus einem Ordner in das aktuelle Dokument eingefügt werden. Der Pfad, in dem nach den Aufgabendateien gesucht wird kann durch die Option \option*{aufgaben} für jedes Dokument angepasst werden.

\begin{options}
	\keyval{aufgaben}{pfad}\Default{ngb/afg}
	\sidenote{In zukünftigen Versionen sollte der Pfad für die Sammlung zentral anpassbar sein.}
	Setzt den Pfad zur Aufgabensammlung. Als Standard wird in einem Unterordner des Pakets selbst gesucht, von dem aus die Aufgaben jeder \TeX Datei zur Verfügung stehen. Um die Aufgabensammlung an einem anderen Ort zu speicher muss (derzeit) der Pfad in jeder Datei über diese Option angegeben werden.
	
	Der Pfad kann absolut sein (z.B. \texttt{~/user/tex/Aufgaben}) oder eine relative Pfadangabe (z.B. \texttt{../../Aufgaben}).
\end{options}

\begin{commands}
	\command{aufgabeLaden}[\marg{name}]
	Lädt eine Aufgabe aus der Aufgabensammlung, wenn sie vorhanden ist. Die Datei muss den Namen \texttt{\meta{name}.afg.tex} haben und im Ordner der Aufgabensammlung liegen.
	
	Ohne Angabe eines neuen Pfades für die Aufgabensammlung
	ist der zusammengesetzte Pfad für eine Aufgabe dann also
	\texttt{ngb/afg/\meta{name}.afg.tex}.
	
	Es können auch Unterordner angegeben werden. (Z.B. \cs*{aufgabeLaden}\Marg{if/aufgabe-1}. Die
	Aufgabendatei liegt in dem Fall also unter \texttt{ngb/afg/if/aufgabe-1.afg.tex}.)
\end{commands}


\section{Materialsammlung}\label{sec:materialsammlung}
Das Paket liefert eine Reihe von Materialien mit, die (hauptsächlich bezogen auf die Lehrpläne in NRW) häufig in Dokumenten auftauchen. Die einzelnen Fächer stellen Makros bereit, um Fachspezifische Materialien nachzuladen. Dies ist vor allem für die Informatik relevant, für das vorgefertigte \emph{UML-Diagramme}, \emph{Struktogramme} und \emph{Quelltexte} eingebunden werden können.

Details dazu sind bei der Beschreibung der Fächer unter \prettyref{sec:materialinfo} und \prettyref{sec:materialmathe} beschrieben.

Generell sollten die Materialdateien möglichst so gehalten sein, dass sie auch ohne das Paket \ngbschule\ verwendet werden können (Struktogramme sollten also zum Beispiel nicht die Makros \cs*{Anweisung} etc. verwenden, sondern direkt die Makros aus dem \pkg*{stuktex}-Paket).