% !TeX root = ngbschule-doc
\part{Fächer}\label{part:faecher}

\section{Fach Informatik}\label{sec:fachinfo}

\subsection{Quelltexte in Tabellen}
...

\subsection{Materialien Informatik}\label{sec:materialinfo}
\sidenote{Sobald für mich relevant werden auch Sequenzdiagramme und PAPs ergänzt.}
Für das Fach Informatik lassen sich auf einfache Weise eine Reihe vorgefertigte UML-Diagramme, Struktogramme und Quelltexte einbinden.

\begin{commands}
	\command{cd}[\marg{name}]
	Fügt das Klassendiagramm mit dem Namen \meta{name} in das Dokument ein.
	
	\command{lst}[\marg{name}]
	Fügt die Quelltextdatei mit dem Namen \meta{name} in das Dokument ein.
	
	\command{nss}[\marg{name}]
	Fügt das Struktogramm mit dem Namen \meta{name} in das Dokument ein.
\end{commands}

Die Namen der Materialien halten sich an die Punktnotation, um eine möglichst einfache Organisation zu gewährleisten. Die Namen lassen sich so auch mehrfach verwenden, um zum Beispiel verschiedene Darstellungen eines Algorithmus darzustellen.

Die Operation \texttt{pop} des \texttt{Stack} ist zum Beispiel unter dem Namen \texttt{stack.pop} gespeichert. Mit \cs*{lst}\Marg{stack.pop} wird eine Quelltextdarstellung der Operation eingebunden (Standard: Java), während \cs*{nss}\Marg{stack.pop} das entsprechende Struktogramm anzeigt. \cs*{cd}\Marg{stack} bindet wiederum ein Klassendiagramm des \texttt{Stack} ein und \cs*{lst}\Marg{stack.pop.pseudo} einen Pseudocode.

Alle Dateien für das Fach Informatik liegen im Unterordner \texttt{if}.Die Dateinamen setzen sich aus dem Namen des Materials und einer Endung pro Typ zusammen (abgeleitet aus dem Namen des Makros, z.B. \texttt{.cd.tex} für \cs*{cd}).

\subsubsection{UML-Diagramme}
Klassen- und Objektdiagramme liegen in Dateien mit der Endung \texttt{.cd.tex} im Materialordner. \\Zum Beispiel \texttt{ngb/inc/if/stack.cd.tex}. Dabei ist zu beachten, dass die Namen, die Operationen/Methoden entsprechen, in 
der Regel Objektdiagramme enthalten, die die Veränderungen der Zeiger beim Aufruf der Operation darstellen.

\begin{multicols}{2}\ttfamily
	\begin{smallitemize}
		\item stack
		\item stack.pop
		\item stack.push
		\item queue
		\item queue.enqueue
		\item queue.dequeue
		\item list
		\item list.insert
		\item list.append
		\item list.delete
	\end{smallitemize}
\end{multicols}

\subsubsection{Quelltexte}
Quelltexte liegen in Dateien mit der Endung \texttt{.lst.tex} im Materialordner. \\Zum Beispiel \texttt{ngb/inc/if/stack.pop.lst.tex}.
Bei mehreren verfügbaren Sprachen wird der Name der Sprache angehängt, also zum Beispiel \texttt{stack.pop.java.lst.tex}. Das Makro
sucht zunächst nach einer Datei mit dem exakten angegebenen Namen und als Fallback nach der Java-Version.

\begin{multicols}{2}\ttfamily
	\begin{smallitemize}
		\item stack.pop.java
		\item stack.push.java
		\item queue.enqueue.java
		\item queue.dequeue.java
		\item list.insert.java
		\item list.append.java
		\item list.delete.java
		\item stack.pop.pseudo
		\item stack.push.pseudo
		\item queue.enqueue.pseudo
		\item queue.dequeue.pseudo
		\item list.insert.pseudo
		\item list.append.pseudo
		\item list.delete.pseudo
		\item stack.pop.python
		\item stack.push.python
		\item queue.enqueue.python
		\item queue.dequeue.python
		\item list.insert.python
		\item list.append.python
		\item list.delete.python
	\end{smallitemize}
\end{multicols}

\subsubsection{Struktogramme}
Struktogramme liegen in Dateien mit der Endung \texttt{.nss} im Materialordner. \\Zum Beispiel \texttt{ngb/inc/if/stack.pop.nss}.

\begin{multicols}{2}\ttfamily
	\begin{smallitemize}
		\item stack.pop
		\item stack.push
		\item queue.enqueue
		\item queue.dequeue
		\item list.insert
		\item list.append
		\item list.delete
	\end{smallitemize}
\end{multicols}

\section{Fach Mathematik}\label{sec:fachmathe}

Für Dokumente mit dem Typ \enquote{ab} (\keyis*-{typ}{ab}) werden weitere Pakete nachgeladen:
\begin{multicols}{2}
	\begin{smallitemize}
		\item \pkg{amsmath}
		\item \pkg{amssymb}
		\item \pkg{amstext}
		\item \pkg{amscd}
		\item \pkg{exscale}
		\item \pkg{tkz-base}
		\item \pkg{tkz-euclide} und \cs*{usetkzobj}\Marg{all} wird ausgeführt.
		\item \pkg{tkz-fct} benötigt eine Installation von \texttt{gnuplot} und \texttt{pdflatex} muss mit dem Schalter \texttt{--shell-escape} gestartet werden (\zB\  \texttt{pdflatex --shell-escape "ab-mit-tkz.tex"}).
		\item \pkg{skmath} mit den Optionen \option*{commonsets} und \keyis*-{notation}{german}.
		\item \pkg{units}
		\item \pkg{gauss}
	\end{smallitemize}
\end{multicols}

Die Dokumentation der Pakete hilft bei der Verwendung, aber gerade für die Geometrie-Pakete (\pkg*{tkz}) werden viele hilfreiche Kommandos ergänzt (siehe dazu \ref{sec:geometrie}).

\subsection{Mathematische Erweiterungen}\label{sec:allgemeinmathe}
\begin{environments}
	\environment{sachaufgabe}
	Setzt die Lösung einer Sach-/Textaufgabe mit \cs{frage}, \cs{rechnung} und \cs{antwort}. Die Umgebung hat nur semantische Bedeutung, da sie einfach eine \env*{description} Umgebung darstellt.
	
	\begin{example}
		\begin{sachaufgabe}
			\frage Wieviele Enten schwimmen auf dem See?
			\rechnung $\text{Alle Meine Entchen} - \text{Meine Entchen} = \text{Alle}$
			\antwort \emph{Alle} Entchen schwimmen auf dem See.
		\end{sachaufgabe}
	\end{example}
\end{environments}
\begin{commands}
	\command{grad}[\marg{gradzahl}] Formatiert \meta{gradzahl} mit dem Grad-Symbol ($^\circ$): \grad{36}
	
	\command{punkt}[\sarg\marg{x}\marg{y}\marg{z}] Setzt einen Punkt mit zwei Koordinaten: \punkt{2}{-.5}
	
	Die Optionale Sternvariante setzt einen Punkt mit drei Koordinaten: \punkt*{2}{-.5}{\dfrac{1}{2}}
	
	\command{bs} Erzeugt einen backslash: $\bs$
	
	\command{qed} Erzeugt eine Box als Abschluss von Beweisen: \qed
\end{commands}

\subsection{Geometrische Konstruktionen}\label{sec:geometrie}

\subsection{Materialien Mathematik}\label{sec:materialmathe}
Für Mathematik sind vor allem geometrische Konstruktionen und Merksätze / Definitionen angedacht.

\subsection{Hinweise zum Formelsatz}\label{sec:formelsatz}
Um die Darstellung von Formeln möglichst einheitlich zu gestalten, werden im Folgenden die verschiedenen mathematischen Umgebungen verglichen und Beispiele für wiederkehrende Formeln wie Termumformungen oder Gleichungssysteme gegeben.

\subsubsection*{Mathematische Umgebungen}
\begin{itemize}
	\item equation
	\item array
	\item align
	\item alignat
\end{itemize}

\subsubsection*{Formelsatz}

\paragraph{Gleichungen ausrichten}

Zum Ausrichten von Gleichungen und Formeln bietet sich die \env{align}\sarg Umgebung an. Sie erlaubt die Ausrichtung anhand von \& Zeichen.
\begin{example}
	\begin{align*}
	f'(x) &= 9x^2 + \frac{1}{2x^3} \\
	f'(-1) &= 9\cdot -1^2 + \frac{1}{2\cdot -1^3} \\
		&= 9\cdot 1 + \frac{1}{2\cdot -1} \\
		&= 8,5
	\end{align*}
\end{example}


Sollen Äquivalenzumformungen inklusive der Umformungsschritte dargestellt werden, dann sollte die Umgebung \env{alignat}\sarg benutzt werden, die weitaus flexibler ist. Dadurch können die Gleichungen einheitlich am Gleichzeichen ausgerichtet und zusätzlich Äquivalenzpfeile und Umformungen ergänzt werden.
\begin{example}
	\begin{alignat*}{3}
	&& -\frac{4}{x^3} + \frac{1}{2} &= 0 &&\qquad\rvert-\frac{1}{2}\quad\rvert\cdot -1 \\
	\Leftrightarrow\quad && \frac{4}{x^3} &= \frac{1}{2} &&\qquad\rvert\text{Kehrwert}\\
	\Leftrightarrow\quad && \frac{1}{4}x^3 &= 2 &&\qquad\rvert\cdot 4\quad|\sqrt[3]{\quad}\\
	\Leftrightarrow\quad && x &= 2 &&
	\end{alignat*}
\end{example}

\paragraph{Gleichungssysteme und Matrizen}

Für Gleichungssysteme mit Äquivalenzformungen wird das Paket \pkg{gauss} eingebunden. Um ein Gleichungssysteme zu setzen bietet sich die \pkg*{amsmath} Umgebungen \env*{vmatrix} an. Um zusätzlich Äquivalenzformungen darzustellen ergänzt \pkg*{gauss} die Umgebung \env{gmatrix}\oarg{delimtype}, die als optionalen Parameter die Art der \pkg*{amsmath} Matrix erhält (\code{v} hier).
\begin{sidebyside}
	\[ \begin{gmatrix}[v]
	-2x & + 3y &      &= &  4 \\
	 3x & - 2y & + 5z &= &  9 \\
	  x & +  y & + 5z &= & 13
	\end{gmatrix} \]
\end{sidebyside}

Die zugehörige Koeffizientenmatrix kann entsprechend mit der Option \code{p} gesetzt werden.
\begin{sidebyside}
	\[ \begin{gmatrix}[p]
	-2 &  3 & 0 &  4 \\
	 3 & -2 & 5 &  9 \\
	 1 &  1 & 5 & 13
	\end{gmatrix} \]
\end{sidebyside}

Die Umformungen werden mit dem Makro \cs{rowops} eingeleitet und mit den Kommandos \cs{mult}, \cs{add} und \cs{swap} gesetzt.

Die Ausrichtung von aufeinanderfolgenden Gleichungssystemen wird am besten mit \env*{align}\sarg erreicht. Der Umformungspfeil kann mit \cs{rightsquigarrow} gesetzt werden. Damit die Matrizen nicht zu eng aufeinander liegen sollte beim Zeilenumbruch zwischen den Matrizen ein Abstand angegeben werden (z.B. \verbcode!\\[1em]!).

\begin{example}
	\begin{align*}
		&\begin{gmatrix}[v]
		-2x & + 3y &      &= &  4 \\
		3x & - 2y & + 5z &= &  9 \\
		x & +  y & + 5z &= & 13
		\rowops
		\mult{2}{\cdot 2}\add{2}{0}
		\mult{2}{\cdot -3}\add{2}{1}
		\end{gmatrix} \\[1em]
		\rightsquigarrow&\begin{gmatrix}[v]
		& +5y & + 10z & = 30 \\
		& -5y & - 10z & = -30 \\
		x & +y & +5z&  = 13
		\rowops
		\add{0}{1}
		\end{gmatrix}
	\end{align*}
\end{example}

Aufgrund des Platzmangels werden Brüche in Gleichungssystemen und Matrizen besser mit dem Kommando \cs{tfrac} dargestellt:

\begin{sidebyside}
	\[\begin{pmatrix}
	 0 & 0 & \tfrac{1}{8} \\
	 0 & 0 & -\tfrac{3}{4} \\
	 -\tfrac{1}{2} & 0 & \tfrac{2}{6} \\
	\end{pmatrix}\]
\end{sidebyside}

%		Durch Umformungen erhält man:

%		\[ \begin{gmatrix}[v]
%		& +5y & + 10z & = 30 \\
%		& -5y & - 10z & = -30 \\
%		x & +y & +5z&  = 13
%		\rowops
%		\add{0}{1}
%		\end{gmatrix} \]
%		\[ \begin{gmatrix}[v]
%		& +5y & + 10z & = 30 \\
%		& & 0 & = 0 \\
%		x & +y & +5z&  = 13
%		\end{gmatrix} \]