\part{Einleitung}\label{part:einleitung}

\section{Typografie und Farben}\label{sec:typografie}
Die Gestaltung der Dokumente wird zum größten Teil in der Datei \texttt{theme.tex} festgelegt, die per \cs*{input} eingebunden wird.

Derzeit wird für alle Dokumente dieselbe Theme-Datei eingebunden, allerdings ist eine Idee in Zukunft unterschiedliche Themes für Material der Sekundarstufe I und II einzubauen.

\subsection{Schriftarten}\label{sec:schriften}
Als Schriftarten werden \pkg{FiraSans} für Textabsätze und \pkg{tgschola} für Überschriften genutzt. Als Monospace-Font wird \pkg{courier} geladen.

\subsubsection{Überschriften}\label{sec:ueberschriften}
Generell werden die Kommandos zum ändern von Schriftarten des \texttt{KOMA}-Paketes genutzt (\cs*{setkomafont} und \cs*{addtokomafont}), um die Schriftarten anzupassen.

Zusätzlich zu den dort verfügbaren gibt es eine neue Überschrift für den Reihentitel (\texttt{reihe}). Wie alle \texttt{KOMA} Schriften kann sie über \cs*{usekomafont}\Marg{reihe} genutzt werden, was  aber in der Regel von den entsprechenden Makros für Reihe und Titel automatisch übernommen wird.

\subsection{Farben}\label{sec:farben}