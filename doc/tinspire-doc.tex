\documentclass[11pt, a4paper]{scrartcl}

\usepackage[ngerman]{babel}
\usepackage[utf8]{inputenc}
\usepackage[TS1,T1]{fontenc}

\usepackage{fullpage}
\usepackage{tinspire}

\usepackage{xinttools,xintfrac}

\thispagestyle{empty}

\newcounter{loopi}
\newcommand{\makro}[1]{\texttt{\textbackslash #1}}

\begin{document}

\begin{center}\Huge \TIN\end{center}

\section{Tasten}
Für alle Tasten des \TIN stellt das Paket ein eigenes Makro zur verfügung. Die Namen setzen sich aus dem Prefix {\ttfamily TIN} und dem Namen des Taste zusammen. Der Name lässt sich meist aus der Aufschrift der Taste ableiten.

\subsection{Allgemeine Funktionstasten}
Die Funktionstasten links und rechts neben dem Display haben jeweils eigene Namen.
\begin{center}
\begin{tabular}{lc||lc}
	\xintFor #1 in {esc,on,scratch,doc,tab,menu,ctrl,del,
		shift,var}
	\do
	{\addtocounter{loopi}{1}\xintifOdd{\theloopi}
			{\makro{TIN#1}\hspace{1cm} & \csname TIN#1\endcsname\hspace{1cm} & \hspace{1cm}}
			{\makro{TIN#1}\hspace{1cm} & \csname TIN#1\endcsname \\}
	}
	&& \hspace{1cm}\makro{TINenter}\hspace{1cm} & \TINenter \\
\end{tabular}
\end{center}

Alle Funktionstasten mit einer Sekundärfunktion besitzen eine *-Variente, die die zweite Funktion weg lässt.
\begin{center}
\begin{tabular}{lc||lc}
	\makro{TINesc*}\hspace{1cm} &
	\TINesc*\hspace{1cm} & \hspace{1cm}
	\makro{TINon*}\hspace{1cm} & \TINon* \\
	\makro{TINscratch*}\hspace{1cm} &
	\TINscratch*\hspace{1cm} & \hspace{1cm}
	\makro{TINdoc*}\hspace{1cm} & 
	\TINdoc* \\
	\makro{TINtab}\hspace{1cm} &
	\TINtab\hspace{1cm} & \hspace{1cm}
	\makro{TINmenu*}\hspace{1cm} & 
	\TINmenu* \\
	\makro{TINctrl}\hspace{1cm} &
	\TINctrl\hspace{1cm} & \hspace{1cm}
	\makro{TINdel*}\hspace{1cm} & 
	\TINdel* \\
	\makro{TINshift*}\hspace{1cm} &
	\TINshift*\hspace{1cm} & \hspace{1cm}
	\makro{TINvar*}\hspace{1cm} & 
	\TINvar* \\
	&& \hspace{1cm}\makro{TINenter*}\hspace{1cm} & \TINenter* \\
\end{tabular}
\end{center}

\subsection{Nummernblock}
Nummerntasten werden mit dem Makro \makro{TINnum} erstellt. Als Parameter wird die Zahl (0-9) übergeben. Die beiden Funktionstasten für den Dezimalpunkt und die Negation haben wieder eigene Makros.
\begin{center}
	\begin{tabular}{lc||lc||lc}
		\makro{TINnum\{1\}} & \TINnum{1} &
		\makro{TINnum\{2\}} & \TINnum{2} &
		\makro{TINnum\{3\}} & \TINnum{3} \\
		\makro{TINnum\{4\}} & \TINnum{4} &
		\makro{TINnum\{5\}} & \TINnum{5} &
		\makro{TINnum\{6\}} & \TINnum{6} \\
		\makro{TINnum\{7\}} & \TINnum{7} &
		\makro{TINnum\{8\}} & \TINnum{8} &
		\makro{TINnum\{9\}} & \TINnum{9} \\
		\makro{TINnum\{0\}} & \TINnum{0} &
		\makro{TINdot} & \TINdot &
		\makro{TINneg} & \TINneg \\
		&&
		\makro{TINdot*} & \TINdot* &
		\makro{TINneg*} & \TINneg* \\
	\end{tabular}
\end{center}

\subsection{Doppelte Funktionstasten}
Die zweigeteilten Funktionstasten besitzen jeweils drei Makros. Jeweils eines für die linke und rechte Seite, sowie die Kombination beider Tasten zu einer Doppeltaste. Die Namen setzt sich jeweils aus \texttt{TIN} und einem Bezeichner für die Taste zusammen. Das Doppelmakro setzt sich aus dem Prefix und dem Bezeichner der linken gefolgt vom Bezeichner der rechten Taste zusammen.

Beispielsweise werden die Plus-Minus-Tasten über der Enter-Taste mit den Makros \makro{TINplus} und \makro{TINminus} erzeugt. Die Doppeltaste dann mit \makro{TINplusminus}.
\begin{center}
	\begin{tabular}{lc||lc||lc}
		\xintForpair #1#2 in {(eq,trig), (s,lib), (pow,sq), (times,div), (ex,tenx), (plus,minus), (lbr,rbr)}
		\do
		{\makro{TIN#1} & \csname TIN#1\endcsname &
		\makro{TIN#2} & \csname TIN#2\endcsname  &
		\makro{TIN#1#2} & \csname TIN#1#2\endcsname \\
		}
	\end{tabular}
\end{center}

Auch diese Funktionstasten besitzen *-Varianten ohne Sekundärfunktion. Zum Beispiel
\begin{center}
	\begin{tabular}{lc||lc||lc}
		\makro{TINplus*} & \TINplus* &
		\makro{TINminus*} & \TINminus*  &
		\makro{TINplusminus*} & \TINplusminus* \\
	\end{tabular}
\end{center}

\subsection{Buchstabenblock}
Buchstabentasten werden mit dem Makro \makro{TINletter} erstellt. Als Parameter wird die Buchstaben (a-w) übergeben. Die drei unbekannten \texttt{x}, \texttt{y} und \texttt{z} besitzen eigene Makros.
\begin{center}
	\begin{tabular}{lc||lc||lc}
		\makro{TINletter\{a\}} & \TINletter{a} &
		\makro{TINletter\{b\}} & \TINletter{b} &
		\makro{TINletter\{c\}} & \TINletter{c} \\
		&& usw... &&& \\
		\makro{TINx} & \TINx &
		\makro{TINy} & \TINy &
		\makro{TINz} & \TINz \\
	\end{tabular}
\end{center}

Die weiteren Funktionstasten im Buchstabenblock werden wie folgt erzeugt:
\begin{center}
	\begin{tabular}{lc||lc||lc}
		\xintForthree #1#2#3 in {(ee,pi,comma),(symbols,flag,newline)}
		\do
		{\makro{TIN#1} & \csname TIN#1\endcsname &
			\makro{TIN#2} & \csname TIN#2\endcsname  &
			\makro{TIN#3} & \csname TIN#3\endcsname \\
		}
		&&&& \makro{TINspace} & \TINspace \\
	\end{tabular}
\end{center}

\section{Menüs und Eingabefenster}
\subsection{Auswahlmenüs}
Die Umgebung \texttt{tinmenu} kann zum Erzeugen von Menüs verwendet werden, die ähnlich zu denen des \TIN aussehen.
\begin{center}
	\begin{tinmenu}
		\TINmenuitem*{1: Anzeige}
		\TINmenuitem*{2: Analysis}
		\TINselected\TINmenuitem*{3: Algebra}
		\TINmenuitem*{4: Matrizen}
	\end{tinmenu}
\end{center}

\subsection{Eingabefenster}
Mit der Umgebung \texttt{tinwin} können Eingabefenster gezeichnet werden, die ähnlichkeit zu denen des Taschenrechners haben. Die Darstellung wird durch Tabellen erzeugt und kann daher nicht als Bild exportiert werden.
\begin{center}
	\begin{tinwindow}{Wurzeln eines Polynoms}
		\TINwfield{$a_2$=}
		\TINwfield{$a_1$=}
		\TINwfield{$a_0$=}
		\TINwbtns{OK}{Abbruch}
	\end{tinwindow}
\end{center}

\begin{center}
	\begin{tinwin}
		\TINwtitle{Wurzeln eines Polynoms}
		\TINwinput{$a_2$=}{}
		\TINwinput{$a_1$=}{}
		\TINwinput{$a_0$=}{}
	\end{tinwin}
\end{center}

\section{Beispiele}
\subsection{Quadratische Gleichungen lösen}
\begin{enumerate}
	\item \TIN anschalten (\TINon*).
	\item Scratchpad mit \TINscratch* aufrufen.
	\item \TINmenu* aufrufen und \TINnum{3} drücken (\texttin{3: Algebra}).
	\item Erneut Menüpunkt \TINnum{3} wählen (\texttin{3: Polynomwerkzeuge}).
	\item Dann einmal \TINenter* betätigen (oder \TINnum{1} drücken) um die Funktion \texttin{1: Wurzeln eines Polynoms finden...} aufzurufen.
	\item Im Dialogfenster sollten die passenden Einstellungen schon gewählt sein:
	\begin{center}
		\begin{tinwindow}{Wurzeln eines Polynoms finden}
			\TINwfield[2]{Grad}
			\TINwfield[Reell]{Wurzeln}
			\TINwbtns{OK}{Abbruch}
		\end{tinwindow}
	\end{center}
	\item Nach dem Betätigen des OK-Buttons erscheint das Fenster zur Eingabe der Koeffizienten $a$, $b$ und $c$, bzw. hier $a_2$, $a_1$ und $a_0$:
	\begin{center}
		\begin{tinwindow}{Wurzeln eines Polynoms}
			\TINwfield{$a_2$=}
			\TINwfield{$a_1$=}
			\TINwfield{$a_0$=}
			\TINwbtns{OK}{Abbruch}
		\end{tinwindow}
	\end{center}
\end{enumerate}

\subsection{Cheatsheet}
\begin{tabular}{|c|p{8cm}|} \hline
	\TINon* & Einschalten des Taschenrechners. \\ \hline
	\TINscratch* & Aufrufen des \emph{Scratchpads} für schnelle Berechnungen. Ein weiterer Druck wechselt zum Plot-Fenster, in dem schnell Funktionsgraphen geplottet werden können. Jeder weitere Druck wechselt zwischen Rechner und Plot hin und her. \\ \hline
	\TINmenu* & Aufruf des Taschenrechner-Menüs, welches Kontextabhängig Zugriff auf verschiedene Funktionen gibt. \\ \hline
\end{tabular}

\end{document}
