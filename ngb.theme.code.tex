
%% -----------------
%%
%% Ein 'theme' zur Dokumentgestaltung (Schriftarten, Farben, ...)
%%
%% -----------------
%\setlength{\parindent}{0em}
%\setlength{\parskip}{2ex plus0.5ex minus0.5ex}
\setlength{\parskip}{1ex plus0.5ex minus0.9ex}

% Schriftarten
%% Standard Textkörper (Serifenlos)
%\usepackage[familydefault,light]{Chivo}
\usepackage[sfdefault,light]{FiraSans}
%\usepackage[sfdefault,light]{roboto}

%% Überschriften (mit Serifen)
\usepackage{tgschola}
\addtokomafont{disposition}{\rmfamily}

%% Monospace font
\usepackage{courier}

%% Math font
%\usepackage{newtxsf}

\usepackage[T1]{fontenc}
\renewcommand*\oldstylenums[1]{{\firaoldstyle #1}}

% Zeilenabstand anpassen
% https://texwelt.de/wissen/fragen/3/wie-stelle-ich-einen-zeilenabstand-von-15-ein
\usepackage{setspace}
\usepackage{scrhack}
\onehalfspacing

% Footer
% \ifoot{\lizenzSymbol}
% \cfoot{\tiny\lizenzNameKurz}
% \ofoot{\tiny\Autor, Version 0.1}


% Farben
%% Allgemein
\definecolor{ngb.reihe.text}{gray}{0.33}

% Koma Schriften
\addtokomafont{title}{\Large}
%\addtokomafont{reihe}{\normalsize}

%% Informatik
\ifdefstring{\schule@fach}{Informatik}{
	\definecolor{ngb.syntax.hg}{gray}{0.98}
	\lstset{
		frame=single,
		backgroundcolor=\color{ngb.syntax.hg}
	}
}{}